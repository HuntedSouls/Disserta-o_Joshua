\section{Why games?}

Nowadays we live surrounded by games. There are games in smart-phones that are played while we come and go to our everyday tasks, games in our houses, in computers and even where there seems to be no games there are people talking about games. One can guess that, if games are so entwined in our daily lives, there will be a great understanding of them, of what games are, how games are made, what means to play a game. Surprisingly that is not true. Studies in games have a great amount of disagreement in many aspects of a game. To begin with there is no commonly accepted definition of the term game. 

Many researchers addressed the problem of finding a definition \citep{jarvinen2009games,salen2004rules,crawford1984art,schell2014art,juul2010game} but they could not agree in a single definition. With each of them perusing very different paths in their search for understanding each defined games in their own way. Although there is some common ground between some of those definitions. \cite{salen2004rules} systematically compared most definitions from before their work and concluded there was no true agreement between them. \cite{wittgenstein_philosophical_2009} went further to state that there is no possible definition, ``game is a concept with blurred edges''. Such is the hardship of finding a definition for games.

Other aspects of games share this multitude of opinions. Game design or how to make games is a subject that presents a variety of ideas and models for its understanding. Although it has more common parts and shared ideas through many researches this subject is far from a global complete knowledge. There are many other questions pertaining games that remain without a completely accepted answer. Such as `why do people play games?', `how people react to games?', `how people behave in games?'.

In light of such confusion and dissent. The purpose of work is to try to get somewhere further in the way of understanding what are games made of. Of which parts compose this complex and intriguing whole. Hopefully this will bring us closer to completely understanding games. At least it should provide a tool for researchers and designers that wish to adventure in the games domain to start to understand what it is. Therefore, this is an attempt to create a boardgame ontology using the ontological principles of UFO \citep{guizzardi_ontological_2005}. It will provide foundation for a human oriented ontology. This ontology is structured upon the MDA framework \citep{Hunicke2004} as it is simple and widely accepted. 
