\section{Structure of this work}

This first chapter is used to introduce the general idea of this dissertation as well as some important concepts that are used in it.

Chapter 2 will present a literature review, which will cover the topics of this dissertation. First it presents works that are similar to some degree to the one that will be done here. Ontologies made about the game domain, including some that were not about game specifically but about specific aspects of games. Then it establish references used as foundation knowledge for this work, books, theories and ideas that were fundamental to the author understanding of the area. Finally it introduces a review of the subjects nested in the MDA framework. That is, provides a review of works done with respect to mechanics, dynamics and aesthetics that were in accord with their definition.

The next chapter will discourse about the methodology chosen for this work, presenting and explaining the UFO and OntoUML and how they will be used to bring fruition to this ontology. Explaining the ideas of UFO and the reasoning behind the choice of it, about why this is different from other attempts on building ontologies in the games domain. Furthermore it shows why OntoUML is a natural choice when describing an ontology founded in UFO. 

Following the methodology chapter 4 present the results of the research on dynamics. Using this results it is also established a vocabulary of dynamics. Which is used to create the dynamics section of this ontology.

Using the background built until then, the fifth chapter summarizes the proposal of this dissertation. Thoroughly explaining what is to be done in the creation of ontology of boardgames and why is it done so. It provides the theoretical foundation of the ontology, the core modeling choices made. Their explanation and reasoning and how to use the OntoUML language to represent the concepts contained in the ontology.

Chapter 6 will then contain the implementation of the ontology in the prolog language. This is the computable form of the ontology. And this chapter covers its creation as well as functionalities and how to use it. Also covering how to evaluate improvements to this ontology.

Finally chapter 7 concludes this work. Discoursing and summarizing what was done here. Altogether exposing how future works can improve the line of research and even create new ones.

