\section{Game ontologies}

The whole idea of creating a ontology about games is not a novelty. There are many attempts to create such artifact, although not specifically about boardgames, each of them with their own purpose. The most known of those attempts is the Game Ontology Project (GOP) \citep{wiki:gop}, “a framework for describing, analyzing and studying games” \citep{Zagal:2008:GOP}. It  provides  a  structure  to  study  games  elements  based  on four top-level elements:  interface,  rules,  entity manipulation and goals. It  is  a  collaborative  work,  open  to  contributions, although as of 11/11/2019 its development stalled. Some of its elements can be identified in the ontology proposed in this work. Most of it in the Mechanics ontology. Those semblances are further covered in \citep{kritz_buildingOntology}. Roman,  Sandu,  and  Buraga  constructed  an  ontology  for  role-playing games (RPG) in a work inspired by GOP \citep{roman2011owl}.That work uses the OWL language to describe the ontology and intends to facilitate designers’ activities like character creation, NPC generation, simple battle system configuration, etc. A slightly more comprehensive  ontology was created in the realm of RPG \citep{dhuric2015specific}. Although  using  a  specific  game,  \cite{manaworld}, as the source of concepts, the authors claim that the resulting ontology is applicable to massively multiplayer online role-playing games (MMORPG).

More recently, GOP and MDA were also used to provide a model for innovation in digital games \citep{innov:gop:mda}. Three Hundred Mechanics is a video game oriented catalog of game mechanics, with examples \citep{300gm}. It is very comprehensive and provides five collections: Comp-grid, Procedural, Tactics, Tiny Crawl and Misc. It is maintained by a single person, but also seems to have ceased to evolve, as of 2019. In \cite{leon_z._ontology_2010} a Mobile games ontology is developed while investigating the use of UML and OCL for ontology representation. Another recent work focused in video games is \cite{parkkila_ontology_2017}, in which the authors claim that can enable interoperability between games. A Game Character Ontology developed in \cite{sacco_game_2017} provides tools to be used in the creation of game characters.

In the realm of Serious Games there are some other attempts on ontology creation. A group of researchers at Aristotle University of Thessaloniki created an ontology for exergames proposing a "unified model for the semantic representation of exergames" \citep{bamparopoulos_towards_2016} with the objective to propose standards for the research on exergames. In a different approach Tang and Hanneghan created the Game Content Model (GCM) \citep{tang_game_2011}, which is an ontology about documenting serious game design. With the objective of helping new game designers with methods and game design models. Although the main subject of GCM is not exactly the game, but game documentation, it provides great insight in taking the player into account in its model. More specifically for games in education \cite{ghannem2011defining} created a ontology for integrating games and learning processes.