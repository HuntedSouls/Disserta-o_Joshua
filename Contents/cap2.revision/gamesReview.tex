\section{Background knowledge of games}

Although not defining the term game, it is important to be clear about how we understand games. About what are the paving stones for the knowledge used in this work. In such light this section discourses about many authors and their works, which compose the foundation of the view of games domain. First it is important to note that most of the references mentioned here are not specific about boardgames, some are even exclusively about video games, but nonetheless they are important to clarify the way this work look and delve into boardgames. As most of the knowledge in such references are about games in general their ideas are completely valid for boardgames. Although the specifics of how to fit these ideas in this domain was brought mostly by experience.

The most notable foundation is the MDA framework \cite{Hunicke2004}, chosen as the core structure of this dissertation. It will be explained in a section of its own given how important it is to this work.

Close to the MDA framework, but different in the way it relates the game components, is \cite{schell2014art} framework. He defines mechanics much like the MDA and aesthetics in a different but similar way. It includes the visual and sensory aspects of the game. But do not define the emotions as aesthetics and actually relates them as cause-effect. Schell does not include the dynamics in his framework, but it introduces two components, Story and technology. The former is important to understand how the narrative and the reasoning behind mechanics, impact the game. Technology is of great importance here. In Schell's framework it is the actual choice of implementation of the game. How it exists as an artifact, if it is paper and plastic objects, a book, code and data in a digital setting. The importance of it to games is that it imposes limits to the game. An example, for a game to be created for a mobile platform, it cannot have the same quality of graphics and sound as if it was to be played in a complete desktop. In a more complex comparison, Role Playing Games, which are written in books, can have a flexible set of rules, most of the time the rules are dictated by a master and aren't even written on the book. Such characteristic is difficult to obtain in digital games, as these rules needs to be pre-coded. This distinction given to the choice of technology is definitive for this work, as it is why I think worth to study a specific type of technological limitation. Of physical papers and plastic objects and a rulebook, this scope allows to better enlighten this limitations and advantages of a specific technology.

When studying games there are many facets that can be persecuted, \cite{salen2004rules} acknowledge it in their work and even used this in their framework. They termed this process of focusing in a specific facet as framing the game. According to \cite{salen2004rules} there is three possible ways to frame games, as Rules, Play and Culture, each of them focus on certain specific aspects of a game in detriment of other ones. To frame games as Rules means looking into the game as in and of itself without considering who plays the game, neither in what circumstances it happens. According to the authors ``Rules is a \textit{formal} primary schema, and focuses on the intrinsic mathematical structures of games''\citep{salen2004rules}. The Rules frame focus on the mechanics of the games. Increasing the scope of the frame there is Play, this frame begins to take into account the player of the game, what he does and how he behave, as the authors state ``Play is an \textit{experiential} primary schema and emphasizes the player's interaction with the game and other players.'' \cite{salen2004rules}. In accord with the MDA, the Play frame focus on Dynamics and the Aesthetics of the game. The broader scope of framing is Culture, it goes beyond the game and its players to look into the social behaviour and culture surrounding the game, to the authors ``Culture is a \textit{contextual} primary schema, and highlights the cultural contexts into which any game is embedded.'' \cite{salen2004rules}. This framing is not encompassed by the MDA framework, it focus on the environment in which the game exists and how it interacts with the game, not with the game itself as does the MDA. This is an important distinction, for it provides a clear understanding of one of the boundaries of this work's scope. By using the MDA as our foundation we abide to remain in its frame, which means this work frames games in a conjunction of the Rules and Play frames.

