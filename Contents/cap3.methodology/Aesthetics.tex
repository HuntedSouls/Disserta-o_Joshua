\section{Aesthetics}

The aesthetics ontology domain is centered at emotions, as such this ontology is created with foundation in emotions theories. In this light this ontology is based in two theories, the \cite{dillon_way_2010} 6-11 model, which is a theory of emotions in games, and the one presented in \cite{ekmans_atlas} Atlas of Emotions. This last theory was built in a psychological view with no direct relation to games. The objective of using both theories in union is to benefit from the detailed perspective of Ekman's approach, which brings a rich understanding of how emotions function, while using Dillon's simpler approach to appropriately connect the whole theory into the games domain.

Fusing both theories is not inconceivable because both precepts the existence of basic emotions. Their understanding of basic emotions although slightly different has a lot of overlaps and is totally compatible. That said Ekman's model is adopted for the emotions, as it is far more detailed and complete, in detriment of the 6-11 model. From Dillon's model we harness the instincts section, it provides insight in how emotions behave in games. 

Established those directives. The particular method for modeling emotions and instincts is broadly described in the following subsections.

\subsection{The modeling of emotions}

OntoBG-A is centered around the five basic emotions found in the atlas of emotions. The goal of OntoBG is to express boardgames in a way that designers can expand their knowledge about their games. As such this ontology features the emotions states of the atlas as subkinds of the basic emotions. This is due to some advantages these concepts introduce in the ontology. First when analyzing games the broadness of the emotions can create confusion or ambiguity whilst using emotional states specificity provides more clarity in how the game affects the player. It is due to this specificity of the states that when relating to instincts the user of the model can pinpoint more precisely how that instinct interacts with the game. Also emotional states not being exhaustive abides to one of the goals of OntoBG, which is to be expansible according to the needs of the designer. Opposing the emotions that are static for the theory, thus using only them the model would become static as well. With this the designer can include new emotional states he identifies in his game giving him the tool he needs to understand the emotions behaviour in his game.

Composite emotions are acknowledged in the atlas of emotions as it is possible for a given person feel different emotions simultaneously \citep{ekman_are_basic_emotions_nodate}. When this event occurs the union of emotions provokes different reactions and thus it is useful to include such concept in OntoBG-A. But Ekman's model does not explicate any composite emotion thus the inclusion of this concept in this model come as a category with no kind as an example. With this the user of the model that needs to analyze a particular composite emotion can introduce it as a specification of this category and identify which emotions compound it by parthood relations.

Instincts feature as kinds connected to the emotions according to \citeauthor{dillon_way_2010}, using the needed adaptations of disposed emotions. Relations from instincts to emotions should not be viewed as unique and static, as relating instincts to emotional states is also possible and encouraged. But for simplicity we stick to the theoretical framework used as basis.