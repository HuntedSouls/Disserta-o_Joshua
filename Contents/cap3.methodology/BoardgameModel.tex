\section{How to model board games} 

The perspective of modeling board games in accord with UFO is not unique neither simple. As with any modeling activity, there are many possible interpretations of the concepts of the domain, which means that a focus is needed. This focus is to make the choices of interpretations giving more importance to certain aspects of the concepts, with some specific goal in mind. As example, one could model board games as a social activity, focusing on player interactions and consequences of play, another perspective is to model the physical artifact, focusing on the materials and forms used in board game elements.

This ontology focus its lenses in a MDA model perspective, that is, it looks to board games as composition of 3 concepts. That gives a broader spectrum for the perspective of OntoBG, which means it comprehends the artifact, the play of the game as well as the consequences of play. But most important is that MDA looks into games as a construct, that is, it focuses on the fact that it is made with intention, during the games' design. Even when looking on the play of the game or the aftereffects of it, it does so considering them as part of the constructed entity. That said, is important to note that all the modeling choices done in this work take this perspective into account to be faithful to the MDA basis. 



Addressing the modeling directives used to create this ontology is important to be sure that all the choices are reasonable with each other. This should clearly state the view of the world represented by this model. OntoBG uses the following directives for modeling board games:

\begin{itemize}
    \item Mechanics, dynamics and aesthetics are viewed as kinds, as well as mandatory parts of board games
    \item Concepts to be included need to be present on a given board game, that is, concepts exclusive of digital games are not included.
    \item Relationships between mechanics and dynamics or dynamics and aesthetics maintain the same meaning as the MDA framework
    \item Concepts that are neither a mechanic, dynamic or aesthetic cannot feature in the ontology
\end{itemize}

OntoBG will be created as a lightweight ontology. That means it does not provide an axiomatization of the domain. Not being strict on concepts definitions is essential when creating such a preliminary work. Especially when addressing a domain such as games, which are defined through common knowledge rather than scientifically. This also increases the possibility for contribution on expanding the ontology as any one with experience can provide valuable ideas. Not being comprehensive, this work' ontology needs to be enhanced through peer contributions to cover more thoroughly the board game domain.
