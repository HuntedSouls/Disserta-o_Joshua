\section{Dynamics}
Dynamics might be the most complicated domain to model in this ontology, using the definition provided by the MDA framework dynamics become almost about everything. In an essay \cite{leblanc2006tools}, one of the creators of MDA, states ``When we veiw a game in terms of its dynamics, we are asking, 'What happens when the game is played?'`` and the complete answer to this question when we are looking into the whole domain of boardgames becomes overwhelming as there is a multitude of things happening during a single gameplay, there is even more about all possible plays of all possible games. 

Another difficulty found is the lack of specific research on dynamics of games. Although many authors acknowledge and speak of dynamics they do so when focusing their efforts in other aspects of games. So they do not convey a definition of the term neither a good pool of examples. With this there is nowhere to find a big set of terms and concepts with a structure in dynamics of games like it was done with mechanics and aesthetics.

To workaround this troublesome situation this work brings about a quantitative and qualitative research to provide an initial knowledge of dynamics of boardgames. This is done by using a focus group composed of game designers to evaluate on names and concepts of dynamics. Following with a survey applied more widely to validate the results of the focus group and provide more suggestions. The resulting set of concepts will be then evaluated through the survey results and pruned when needed, after such filtering they will be used to compose the ontology of dynamics.

\subsection{The focus group}

A focus group is a method to generate data based on individual experience and the discussion of such individuals on those experiences. Those individuals should have common expertise in the topic to be addressed in the experiment. Even better is if the participants have a good amount of knowledge and experience on the subject. According to \cite{jenny_methodologyfocusgroup_1994} the most important data of a focus group is provided by the interaction and discussion of the individuals. Nonetheless a focus group have to remain focused on the topic to be addressed and thus need to be directed correctly to provide better quality data for the reasearch. \citep{liamputtong_focusgroup_2011,rabiee_focus-group_2004,jenny_methodologyfocusgroup_1994}

The selection of individuals for this focus group should, of course, be composed of the target users of this methodology, designers of boardgames. Also it intends to collect data on dynamics, which happens during the play of the game. Thus long time players of boardgames should also be able to provide valuable insight, and are included in this selection. 

Designers and play-testers of the \textit{Casa do Goblin} collective agreed to be our focus group and will endeavour in this activity. They all have experience in the boardgame domain and have interest in OntoBG as a creation or analysis tool. The participating designers were also both new and experienced ones. The ten participants were composed of 5 men and 5 women, from 23 to 40 years old.

Although experts in boardgames, the participants have differing notions of dynamics or no understanding at all of the specifics. As such it is needed to thoroughly explain the dynamics concept of the MDA to them. Then designers participants will brainstorm on names or terms of dynamics they believe pertain to boardgames. Afterwards they will discuss what was proposed on the brainstorm to filter unfitting concepts and to further develop the ones that require extra atention.

\textit{Casa do Goblin}'s participants will be introduced to the dynamics concept using the definition in the MDA model, as well as further explaining of the notion according to this works's fundamentals. To clearly illustrate the concept, the example of \textit{movement} - \textit{run} - \textit{fear}  is used. Also they will be encouraged to compose generic and simple examples like this one to be sure they grasped the meaning. 

The discussion will be directed towards generating words for dynamics, that is, to explicate concepts they believe to be dynamics into simple words. To provide an anchor from where to start the mechanics featured in this OntoBG-M will be fixed on a board in everyone' views. The purpose of the mechanics will be to be used as basis for thinking of dynamics they found in games, that is, looking to the mechanics and thinking which dynamics emerge from each of them.

Provided the initial basis of words and concepts the group then will review the words generated and discuss what they mean, how they present themselves in games and wetter it is or not a dynamic. If needed little adjustments should be made and the words will become concepts to be used in OntoBG-D, that is, become names of dynamics.

Discussion made during the whole process will be acknowledged as possible concepts and relations that are not dynamics names. In example abstract concepts used to structure the dynamics, connections between types of dynamics, maybe even part-whole relations.



\subsection{Survey on dynamics}

At the begining of the survey there will be a short explanation of what is a dynamic of boardgame according to the MDA. This is to be sure all subjects are on the same terms to the concepts. Also for statistical purposes and further analysis there are questions about their relation to the boardgame world (designer, producer, hobbyist) and how long have they known and played boardgames. This distinction comes from the proposal of MDA that the player of the game and the designer have different perspectives of the game. Which can lead to a significant difference in their opinions about dynamics.

The survey will be composed of questions based on the results of the focus group. For each term created the survey will have a 5 point traditional likert scaled questions. Surveyed subjects answer how much they agree or not with the term as a dynamic that actually happens in games. One of its advantages was having the neutral option between agreement or disagreement, what would allow for answers like not knowing or not understanding. For verification purposes, before the actual questions are made the survey requires some information about who is filling it. These questions inform of how the subject is related to boardgames and how frequently he plays boardgames. \citep{devellis2016scale}

To establish the concepts that would feature in the survey a trimming was necessary. Fifty seven concepts, resulting in 57 questions, would lead to a large survey. Large surveys with obligatory questions are less likely to receive answers which could lead to insufficient number of answers. To adjust this the 57 concepts obtained from the focus group will be narrowed. The redundant concepts, which are contained in one another, will be removed. Keeping only the more generic ones. Also, similar concepts which can together be abstracted into a new one that does not feature in the list will be removed as well. Favoring the more general concept. \citep{malhotra2012pesquisaMarketing}

Furthermore if the subject wanted it could suggests new dynamics he felt were missing. And also provide an e-mail address to be further informed about this research if he was interested. 

The actual survey submitted to the public can be found in \autoref{appendix:a1} the Portuguese version and in \autoref{appendix:a2} the English version. It was made and submitted in both languages to achieve a greater number of answers which leads to greater diversity of opinions. The survey was created unsing google forms and conveyed through facebook and whatsapp groups with boardgame interests as well as in the BoardGameGeek forum. The results will be considered closed after a week of being released.

The results collected are expressed and analyzed in chapter 4. There will be some statistical testing on the data. The intent is to decide the best way of using it, whether simply merging the answers from all groups or putting different weights in each. This will be decided through an ANOVA test, which will determine if the data from different groups have any significant discrepancy. If none is found, all answers will be considered equally, in the other case the professionals should be considered to be better quality data, thus given a bigger weight. After this decision the final value for each concept will be evaluated.

Finally, the concepts that will be used to create OntoBG-D will be the ones with a value greater than 3. Those are the concepts which majorly had answers agreeing with them.

\subsection{Developing the ontology}

After acquiring and processing the data it will be used as a baseline for the concepts of the OntoBG-D. Grouping them together, by similarity, or by generalization characteristics. This ontology will bring up a starting notion of how to address dynamics in boardgames. As any part of this ontology it does not intend on being comprehensive or complete. It does then provide some insight on different ways to understand the dynamics collect. That is, the final classification and relations present in the ontology should be viewed as suggestions.

Correlating and classifying the concepts featuring in the OntoBG-D will be done in accordance to the author experience as well as the discussions made during the focal group. Remaining truthful to the idea of comprising the ontology with experience, making it tune in with the designers needs.