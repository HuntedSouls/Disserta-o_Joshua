\chapter{Crafting an Ontology}

This chapter covers how a domain ontology about boardgames was built. The ontology, henceforth called OntoBG, is composed of 3 blocks and constructed upon the theoretical framework of the Universal Foundational Ontology or UFO \citep{guizzardi_ontological_2005}, using the OntoUML modeling language as described in \cite{guizzardi_ontological_2005,guizzardi_ontoUML_2004,guizzardi2015towards}. 

The purpose of separating the ontology into three sub-ontologies is to put together the three ontologies of mechanics, dynamics and aesthetics in a coherent and understandable single ontology, this will expose the structure of the MDA framework used as theoretical principle of this thesis. Each of those sub-ontologies is assembled using a specific methodology. The particulars of their creation will be further explained in the following sections.

For clarity purposes, these three sub-ontologies of the ontology are named OntoBG-M, OntoBG-D and OntoBG-A, respectively for the mechanics, dynamics and aesthetics.

Before introducing the specifics for each sub-ontology of the OntoBG it is important to specify how it can be used to model this particular domain. Due to the broad philosophical foundation of UFO, this subject will be split in two sections. The first addressing this philosophical theory and the general concepts in this model, afterwards it is addressed how it will be applied in the domain of boardgames to give fruition to OntoBG.


\section{Thinking of UFO} 


Games are endurants as far as this ontology is concerned. So UFO-a is used in this work to model such endurant. The interest, then, is to look up to its characteristics as an endurant not a perdurant. There is some aspects of games that can be seen as perdurants, a specific match, the social interaction of games and other ones. But for the MDA framework games are studied as endurants, static things that exist.

First and foremost, an ontology is founded upon concepts of a given domain. To build it we need to acknowledge and understand the reality to be conceptualized. In this case, the more abstract concepts comes from the MDA framework definitions: games are mechanics, dynamics and aesthetics. For the specifics, each will have a different approach, as they are very different in nature. For mechanics it was possible to build the reality to be modeled perusing many authors on game studies as well as widely accepted knowledge bases of board games. Dynamics proved a more difficult subject, it was needed to use research methods to uproot the reality, as there are little specific literature in the area. Emotion models were studied to bring about the aesthetics reality and become a basis for this work's ontology.

Reality provided through the MDA definition of each part implies that mechanics, dynamics and aesthetics are all kinds. Clearly they are sortals, given they provide identity to all parts of the game. Rigidity comes from the necessity of games to have those parts. MDA understands that games are composed of three components, and they are necessary to all games, they define the game. With a big number of kinds in the ontology, for simplicity all are stereotyped as kinds and never as subkinds.

Featuring a big number of concepts, if OntoBG had all possible relationships among those concepts it would become useless, as nobody could understand it. With this in mind, many of such relationships are omitted. But the ones that are in the ontology will be examples of how to express an information in the ontology. Should then be seen as templates to be extended to any other concepts it seems to fit. To help this simplification, the logical restrictions of the model will express how to relate two concepts. More specifically how one cannot relate two individuals or about the correct cardinality for some relationships. 

\subsection{OntoUML}


OntoUML is a language created to express the ontologies created using UFO. His specification are perfectly matched to UFO-a definitions and concepts, which means it is a language to model endurants. It is made similar to UML, hence the name, because of the proximity of UFO and UML standards. Many concepts well established in UML are in accord with UFO, only needing some specifications to be correct. Thus OntoUML is a heavyweight extension to UML to allow UFO standards to be written in UML. 

To make use of this language, this work uses Menthor, a free application developed by \citeauthor{guizzardi_ontological_2005}. Menthor was created through the specifications of OntoUML, it uses these standards to be a tool for creating domain ontologies from UFO. Similarity to UML makes it familiar and easy to use. Restrictions are scripted in OCL. Overall, it is a powerful yet simple tool to create ontologies. Thus its appropriateness for this work is undeniable.

Used to provide visual representation of the model, OntoUML is of great use to understand the data representation and modeling. Thus it is used mainly as a way to convey the structure of OntoBG keeping it understandable and readable.
\section{How to model board games} 

The perspective of modeling board games in accord with UFO is not unique neither simple. As with any modeling activity, there are many possible interpretations of the concepts of the domain, which means that a focus is needed. This focus is to make the choices of interpretations giving more importance to certain aspects of the concepts, with some specific goal in mind. As example, one could model board games as a social activity, focusing on player interactions and consequences of play, another perspective is to model the physical artifact, focusing on the materials and forms used in board game elements.

This ontology focus its lenses in a MDA model perspective, that is, it looks to board games as composition of 3 concepts. That gives a broader spectrum for the perspective of OntoBG, which means it comprehends the artifact, the play of the game as well as the consequences of play. But most important is that MDA looks into games as a construct, that is, it focuses on the fact that it is made with intention, during the games' design. Even when looking on the play of the game or the aftereffects of it, it does so considering them as part of the constructed entity. That said, is important to note that all the modeling choices done in this work take this perspective into account to be faithful to the MDA basis. 



Addressing the modeling directives used to create this ontology is important to be sure that all the choices are reasonable with each other. This should clearly state the view of the world represented by this model. OntoBG uses the following directives for modeling board games:

\begin{itemize}
    \item Mechanics, dynamics and aesthetics are viewed as kinds, as well as mandatory parts of board games
    \item Concepts to be included need to be present on a given board game, that is, concepts exclusive of digital games are not included.
    \item Relationships between mechanics and dynamics or dynamics and aesthetics maintain the same meaning as the MDA framework
    \item Concepts that are neither a mechanic, dynamic or aesthetic cannot feature in the ontology
\end{itemize}

OntoBG will be created as a lightweight ontology. That means it does not provide an axiomatization of the domain. Not being strict on concepts definitions is essential when creating such a preliminary work. Especially when addressing a domain such as games, which are defined through common knowledge rather than scientifically. This also increases the possibility for contribution on expanding the ontology as any one with experience can provide valuable ideas. Not being comprehensive, this work' ontology needs to be enhanced through peer contributions to cover more thoroughly the board game domain.


\section{Mechanics}
The mechanics ontology will be a restructuring of the board game mechanics ontology created previously in \cite{kritz_buildingOntology}. Although it was created using a different methodology it will be used as a foundation for this mechanics ontology. There is a necessity to adapt this ontology and its description to fit in the UFO standards. It will not change much of the structure of the ontology and the main idea will remain untouched. But this will enrich naturally the ontology as the UFO provides a much larger semantic value than the previously used methodology.

To access the situation all of the nodes in the original ontology will be translated as kinds and subkinds. The is-a relations between the nodes will become generalizations as in the OntoUML specifications they are semantically equal.
\section{Dynamics}

Dynamics might be the most complicated domain to model in this ontology. Using the definition provided by the MDA framework dynamics become almost about everything. In an essay \cite{leblanc2006tools}, one of the creators of MDA, states ``When we view a game in terms of its dynamics, we are asking, 'What happens when the game is played?'' and the complete answer to this question when we are looking into the whole domain of board games becomes overwhelming as there is a multitude of things happening during a single gameplay, there is even more about all possible plays of all possible games. 

Another difficulty found is the lack of specific research on dynamics of games. Although many authors acknowledge and speak of dynamics they do so when focusing their efforts in other aspects of games. So they do not convey a definition of the term neither a good pool of examples. With this there is nowhere to find a big set of terms and concepts with a structure in dynamics of games like it was done with mechanics and aesthetics.

To workaround this troublesome situation this work brings about a quantitative and qualitative research to provide an initial knowledge of dynamics of board games. This is done by using a focus group composed of game designers to evaluate on names and concepts of dynamics. Following with a survey applied more widely to validate the results of the focus group and provide more suggestions. The resulting set of concepts will be then evaluated through the survey results and pruned when needed, after such filtering they will be used to compose the ontology of dynamics.

\subsection{The focus group}

A focus group is a method to generate data based on individual experience and the discussion of such individuals on those experiences. Those individuals should have common expertise in the topic to be addressed in the experiment. Even better is if the participants have a good amount of knowledge and experience on the subject. According to \cite{jenny_methodologyfocusgroup_1994} the most important data of a focus group is provided by the interaction and discussion of the individuals. Nonetheless a focus group have to remain focused on the topic to be addressed and thus need to be directed correctly to provide better quality data for the reasearch. \citep{liamputtong_focusgroup_2011,rabiee_focus-group_2004,jenny_methodologyfocusgroup_1994}

The selection of individuals for this focus group should, of course, be composed of the target users of this methodology, designers of board games. Also it intends to collect data on dynamics, which happens during the play of the game. Thus long time players of board games should also be able to provide valuable insight, and are included in this selection. 

Designers and play-testers of the \textit{Casa do Goblin} collective agreed to be our focus group and will endeavour in this activity. They all have experience in the board game domain and have interest in OntoBG as a creation or analysis tool. The participating designers were also both new and experienced ones. The ten participants were composed of 5 men and 5 women, from 23 to 40 years old.

Although experts in board games, the participants have differing notions of dynamics or no understanding at all of the specifics. As such it is needed to thoroughly explain the dynamics concept of the MDA to them. Then designers participants will brainstorm on names or terms of dynamics they believe pertain to board games. Afterwards they will discuss what was proposed on the brainstorm to filter unfitting concepts and to further develop the ones that require extra atention.

\textit{Casa do Goblin}'s participants will be introduced to the dynamics concept using the definition in the MDA model, as well as further explaining of the notion according to this works's fundamentals. To clearly illustrate the concept, the example of \textit{movement} - \textit{run} - \textit{fear}  is used. Also they will be encouraged to compose generic and simple examples like this one to be sure they grasped the meaning. 

The discussion will be directed towards generating words for dynamics, that is, to explicate concepts they believe to be dynamics into simple words. To provide an anchor from where to start the mechanics featured in this OntoBG-M will be fixed on a board in everyone' views. The purpose of the mechanics will be to be used as basis for thinking of dynamics they found in games, that is, looking to the mechanics and thinking which dynamics emerge from each of them.

Provided the initial basis of words and concepts the group then will review the words generated and discuss what they mean, how they present themselves in games and wetter it is or not a dynamic. If needed little adjustments should be made and the words will become concepts to be used in OntoBG-D, that is, become names of dynamics.

Discussion made during the whole process will be acknowledged as possible concepts and relations that are not dynamics names. In example abstract concepts used to structure the dynamics, connections between types of dynamics, maybe even part-whole relations.



\subsection{Survey on dynamics}

At the begining of the survey there will be a short explanation of what is a dynamic of board game according to the MDA. This is to be sure all subjects are on the same terms to the concepts. Also for statistical purposes and further analysis there are questions about their relation to the board game world (designer, producer, hobbyist) and how long have they known and played board games. This distinction comes from the proposal of MDA that the player of the game and the designer have different perspectives of the game. Which can lead to a significant difference in their opinions about dynamics.

The survey will be composed of questions based on the results of the focus group. For each term created the survey will have a 5 point traditional likert scaled questions. Surveyed subjects answer how much they agree or not with the term as a dynamic that actually happens in games. One of its advantages was having the neutral option between agreement or disagreement, what would allow for answers like not knowing or not understanding. For verification purposes, before the actual questions are made the survey requires some information about who is filling it. These questions inform of how the subject is related to board games and how frequently he plays board games. \citep{devellis2016scale}

To establish the concepts that would feature in the survey a trimming was necessary. Fifty seven concepts, resulting in 57 questions, would lead to a large survey. Large surveys with obligatory questions are less likely to receive answers which could lead to insufficient number of answers. To adjust this the 57 concepts obtained from the focus group will be narrowed. The redundant concepts, which are contained in one another, will be removed. Keeping only the more generic ones. Also, similar concepts which can together be abstracted into a new one that does not feature in the list will be removed as well. Favoring the more general concept. \citep{malhotra2012pesquisaMarketing}

Furthermore if the subject wanted it could suggests new dynamics he felt were missing. And also provide an e-mail address to be further informed about this research if he was interested. 

The actual survey submitted to the public can be found in \autoref{appendix:a1} the Portuguese version and in \autoref{appendix:a2} the English version. It was made and submitted in both languages to achieve a greater number of answers which leads to greater diversity of opinions. The survey was created unsing google forms and conveyed through facebook and whatsapp groups with board game interests as well as in the board gameGeek forum. The results will be considered closed after a week of being released.

The results collected are expressed and analyzed in chapter 4. There will be some statistical testing on the data. The intent is to decide the best way of using it, whether simply merging the answers from all groups or putting different weights in each. This will be decided through an ANOVA test, which will determine if the data from different groups have any significant discrepancy. If none is found, all answers will be considered equally, in the other case the professionals should be considered to be better quality data, thus given a bigger weight. After this decision the final value for each concept will be evaluated.

Finally, the concepts that will be used to create OntoBG-D will be the ones with a value greater than 3. Those are the concepts which majorly had answers agreeing with them.

\subsection{Developing the ontology}

After acquiring and processing the data it will be used as a baseline for the concepts of the OntoBG-D. Grouping them together, by similarity, or by generalization characteristics. This ontology will bring up a starting notion of how to address dynamics in board games. As any part of this ontology it does not intend on being comprehensive or complete. It does then provide some insight on different ways to understand the dynamics collect. That is, the final classification and relations present in the ontology should be viewed as suggestions.

Correlating and classifying the concepts featuring in the OntoBG-D will be done in accordance to the author experience as well as the discussions made during the focal group. Remaining truthful to the idea of comprising the ontology with experience, making it tune in with the designers needs.
\section{Aesthetics}

The aesthetics ontology domain is centered on emotions, as such this ontology is created with foundation in emotions theories. In this light this ontology is based in two theories, the \cite{dillon_way_2010} 6-11 model, which is a theory of emotions in games, and the one presented in \cite{ekmans_atlas} Atlas of Emotions. This last theory was built in a psychological view with no direct relation to games. The objective of using both theories in union is to benefit from the detailed perspective of Ekman's approach, which brings a rich understanding of how emotions behave, while using Dillon's simpler approach to appropriately connect the whole theory into the games domain.

Fusing both theories is not inconceivable because both precepts the existence of basic emotions. Their understanding of basic emotions although slightly different have a lot of overlaps and are totally compatible. That said, Ekman's model is adopted for the emotions, as it is far more detailed and complete, in detriment of the 6-11 model. From Dillon's model we harness the instincts section, it provides insight in how emotions behave in games. 

Established those directives, the particular method for modeling emotions and instincts is broadly described in the following subsections.

\subsection{The modeling of emotions}

OntoBG-A is centered around the five basic emotions found in the Atlas of Emotions. The goal of OntoBG is to express boardgames in a way that designers can expand their knowledge about their games. As such this ontology features the emotions states of the atlas as subkinds of the basic emotions. This is due to some advantages these concepts introduce in the ontology. First when analyzing games, the broadness of the emotions can create confusion or ambiguity whilst using emotional states specificity provides more clarity in how the game affects the player. It is due to this specificity of the states that when relating to instincts the user of the model can pinpoint more precisely how that instinct interacts with the game. Also emotional states not being exhaustive abides to one of the goals of OntoBG, which is to be expansible according to the needs of the designer. Opposing the emotions that are static for the theory, thus using only them the model would become static as well. With this the designer can include new emotional states he identifies in his game giving him the tool he needs to understand the emotions behaviour in his game.

Composite emotions are acknowledged in the Atlas of Emotions as it is possible for a given person to feel different emotions simultaneously \citep{ekman_are_basic_emotions_nodate}. When this event occurs the union of emotions provokes different reactions and thus it is useful to include such concept in OntoBG-A. But Ekman's model does not explicate any composite emotion thus the inclusion of this concept in this model come as a category with no kind as an example. With this the user of the model that needs to analyze a particular composite emotion can introduce it as a specification of this category and identify which emotions compose it by parthood relationships.

Instincts feature as kinds connected to the emotions according to \citeauthor{dillon_way_2010}, using the needed adaptations of disposed emotions. Relationships from instincts to emotions should not be viewed as unique and static, as relating instincts to emotional states is also possible and encouraged. But for simplicity we stick to the theoretical framework used as basis.
\section{Using the ontology}

The main objective of OntoBG is to help designing and analyzing games. To achieve this it was created a Prolog program implementing the data model, containing the rules for describing games using the OntoBG specification, thus allowing a user to have a database of games. From such database one can extract information he needs from the ontology using the rules in the program.

Prolog was chosen as it is a logical language that is widely used and have great support. Logic is used to express the rules of the ontology. Rules that provide a reasoning which match the knowledge contained in OntoBG. In other words, the rules explain the inherent data of the ontology in system. This system is then used to actual manipulation on the ontology and associated database. 

Querying the base of games can be made using the Prolog commands. Those will bring some information required. Such as, if a game is already in the database, if there is a similar game or even to suggest a variation of a game. By manipulating the data modeled through this program, the user can find useful information to his work. 


\section{Scope delimitation} 

Important reminder is that OntoBG is not comprehensive and do not claims to become so. To map all possible ramifications of games might be impossible. So this ontology shall be seen as a preliminary work towards understanding and modeling the nature of board games. This scope delimitation is not about removing, it is about gaining. With this, OntoBG can be flexible to attain the most simple needs of a researcher or designer while still being able to grow more complex. Reducing it also turns it more palatable for anyone, not familiar with modeling or ontologies in general, to use.

Fixing the scope on board games is important to lean this work. This is important to achieve the objectives of OntoBG. It permits that the ontology can be increased with ease and simplicity. Also bringing consistency to the knowledge it represents, guaranteeing that OntoBG is easy to understand and use.


