\section{Thinking of UFO} 


Games are endurants as far as this ontology is concerned. So UFO-a is used in this work to model such endurant. The interest, then, is to look up to its characteristics as an endurant not a perdurant. There is some aspects of games that can be seen as perdurants, a specific match, the social interaction of games and other ones. But for the MDA framework games are studied as endurants, static things that exists.

First and foremost, an ontology is founded upon concepts of a given domain. To build it we need to acknowledge and understand the reality to be conceptualized. In this case, the more abstract concepts comes from the MDA framework definitions: games are mechanics, dynamics and aesthetics. For the specifics, each will have a different approach, as they are very different in nature. For mechanics it was possible to build the reality to be modeled perusing many authors on game studies as well as widely accepted knowledge bases of board games. Dynamics proved a more difficult subject, it was needed to use research methods to uproot the reality, as there are little specific literature in the area. Emotion models were studied to bring about the aesthetics reality and become a basis for this work's ontology.

Reality provided through the MDA definition of each part implies that mechanics, dynamics and aesthetics are all kinds. Clearly they are sortals, given they provide identity to all parts of the game. Rigidity comes from the necessity of games to have those parts. MDA understand games are composed of three components, and they are necessary to all games, they define the game. With a big number of kinds in the ontology, for simplicity all are stereotyped as kinds and never as subkinds.

Featuring a big number of concepts, if OntoBG had all possible relationships among those concepts it would become useless, as nobody could understand it. With this in mind, many such relationships are omitted. But the ones that are in the ontology will be examples of how to express a information in the ontology. Should then be seen as templates to be extended to any other concepts it seem fit. To help this simplification, the logical restrictions of the model will express how to relate two concepts. More specifically how one cannot relate two individuals or about the correct cardinality for some relations. 

\subsection{OntoUML}


OntoUML is a language created to express the ontologies created using UFO. Its specification are perfectly matched to UFO-a definitions and concepts, which means it is a language to model endurants. It is made similar to UML, hence the name, because of the proximity of UFO and UML standards. Many concepts well established in UML are in accord to UFO, only needing some specifications to be correct. Thus OntoUML is a heavyweight extension to UML to allow UFO standards to be written in UML. 

To make use of this language, this work uses Menthor, a free application developed by \citeauthor{guizzardi_ontological_2005}. Menthor was created through the specifications of OntoUML, it uses this standards to be a tool for creating domain ontologies from UFO. Similarity to UML make it familiar and easy to use. Restrictions are scripted in OCL. Overall, it is a powerful yet simple tool to create ontologies. Thus its appropriateness for this work is undeniable.

Used to provide visual representation of the model, OntoUML is of great use to understand the data representation and modeling. Thus its used mainly as a way to convey the structure of OntoBG keeping it understandable and readable.