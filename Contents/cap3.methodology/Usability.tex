\section{Using the ontology}

The main objective of OntoBG is to help designing and analyzing games. To achieve this it was created a Prolog program implementing the data model. Containing the rules for describing games using the OntoBG specification. Thus allowing a user to have a database of games. From such database he can extract information he needs from the ontology using the rules in the program.

Prolog was chosen as it is a logical language that is widely used and have great support. Logic is used to express the rules of the ontology. Rules that provide a reasoning which match the knowledge contained in OntoBG. On other words, the rules explicates the inherent data of the ontology in system. This system is then used to actual manipulation on the ontology and associated database. 

Querying the base of games can be made using the prolog commands. Those will bring some information required. Such as, if a game is already in the database, if there is a similar game or even to suggest a variation of a game. By manipulating the data modeled through this program, the user can find useful information to his work. 