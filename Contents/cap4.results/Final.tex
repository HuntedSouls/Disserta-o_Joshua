\section{To the survivors, the spoils!}

After consideration on the analysis' result we concluded that averaging both our populations as equals was the best course of action.

With this, a final average level for each concept was calculated. The goal was  extracting from the original list the core game dynamics experienced in boardgames, making the assumption that they are represented by concepts mostly agreed upon on the survey. 

Those with average equal to 3 or higher formed this first vocabulary. To put concepts together in this effort they also needed to be synthesized in a word, since vocabularies are made of words that are linked to a meaning. With that in mind, here is the vocabulary created, composed of 29 of the 40 concepts featured in the survey:

\begin{multicols}{2}
    
\begin{itemize}
    \item Sacrifice: Sacrifice a piece or position for greater gains
    \item Indirect Effect: Execute something when wanting a consequence of its events
    \item Acquire information: Use an action to discover 
    \item Reduce Options: Using an action to reduce other players options
    \item Resource Extinction: Reduce the source of a limited resource or make it useless
    \item Deduction: Use open information to discover hidden information
    \item Game state Change: Use an action to purposely provoke a change in the game
    \item Combo: Chain automatic effects of the game
    \item Blocking: Block another player's action, strategy, progress
    \item One versus all: When one player attacks all the other simultaneously 
    \item All versus one: All players of the game, but one, join forces to defeat the other player
    \item Alliance: When players join forces to achieve mutual benefit
    \item Forceful interpretation: Use a particular point of view to create better benefits for himself
    \item Self objective: Pursue a self appointed objective other than the game's objective
    \item Play safe: Not taking risks, playing only on certainty
    \item Risk play: Accept greater risks seeking greater rewards
    \item Survival: Play only with to avoid elimination
    \item Camping: Stick with a position or action through a lot of time
    \item Protectionism: Protect a specific position or pieces
    \item Action planning: Play accordingly to your next actions, plan a series of actions.
    \item Rush the game: Accelerate the end of the game
    \item Flexible strategy: Change strategy because of the game state
    \item Reject objectives: Intentionally not achieving a game's objective to attain some advantage.
    \item Intimidate: Use of a stronger position to force another player to play as you want
    \item Distraction: Use an action to change other players's attention from your real intention or objective
    \item Small talk: Talking all the time to distract other players
    \item Count resources: Use previous knowledge of the available resources to count them and achieve advantage
    \item Bluffing: Relay false information to manipulate other players actions
    \item Convince: Convince other players
\end{itemize}
\end{multicols}
Of course it is not comprehensive on boardgame dynamics, even more in general game dynamics. Nonetheless it is a starting point, and feature some obvious dynamics as well as non intuitive ones.
