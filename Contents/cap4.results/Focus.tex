\section{Focus group} \todo{CITATIONS!!! ONZE!!!! paper SBGames}

Results from the focus group came in form of knowledge on boardgame dynamics. This knowledge was represented in form of 57 ideas which were proposed and discussed by the group. Analyzing the collected ideas and comprising them into a more compact list was done shortly after. Which was important to do before this ideas could be used in the subsequent tasks. 

Primarily the group brainstormed ideas of dynamics. It developed a lot of concepts, which were then discussed and checked whether they were dynamics or not. Many concepts were mentioned by the same person even if in different terms. This means the group was well aliened in how they understand the dynamics. Other concepts met a big dissension on whether they were dynamics or not. Further studies should be made to evaluate this, thus they were excluded from this work. The group could not associate some ideas, which seemed dynamics, with a proper mechanic. Without such connection they were not included in this ontology. Discussing the concepts proved to be most valuable. Some of them did not appeared to be dynamics in a first glance. But in another perspective or situation they were dynamics.

Apart from the dynamic concepts created the discussion featured their relation to mechanics and aesthetics. As a group, they related many dynamics to the other concepts when explaining them or trying to understand. Also sometimes relating dynamics to each other. Establishing that some are incompatible with each other, as well as generalizations between them.

Finally a list of 40 dynamics was comprised. They can be found at \ref{tab:focusgroupraw}.Those were used to form the questions of the survey. 

