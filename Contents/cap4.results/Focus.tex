\section{Focus group} 

Results from the focus group came in form of knowledge on boardgame dynamics. This knowledge was represented in form of 57 ideas which were proposed and discussed by the group. Analyzing the collected ideas and comprising them into a more compact list was done shortly after. Which was important to do before this ideas could be used in the subsequent tasks. 

Primarily the group proposed ideas of dynamics. It developed a lot of concepts, which were then discussed and checked whether they were dynamics or not. Many concepts were mentioned by more than one person even if in different terms. This means the group was well aligned in how they understand the dynamics. Other concepts met a big dissension on whether they were dynamics or not. Further studies should be made to evaluate this, thus they were excluded from this work. The group could not associate some ideas, which seemed dynamics, with a proper mechanic. Without such connection they were not included in this ontology. Discussing the concepts proved to be most valuable. Some of them did not appeared to be dynamics in a first glance. But in another perspective or situation they were dynamics.

Apart from the dynamic concepts created the discussion featured their relation to mechanics and aesthetics. As a group, they related many dynamics to the other concepts when explaining them or trying to understand. Also sometimes relating dynamics to each other. Establishing that some are incompatible with each other, as well as generalizations between them.

Finally a list of 40 dynamics was comprised. They can be found in the following list.Those were used to form the questions of the survey. 
\begin{multicols}{2}
\begin{itemize}
    \item 1 - Do an action to reduce other players options
    \item 2 - Collateral effect: do A resulting in B intending C.
    \item 3 - Use an action to discover information, see other players reaction.
    \item 4 - Block another player
    \item 5 - Use an action to force a change in the game state (phase, turn, etc)
    \item 6 - One versus all, one player decides to attack all others
    \item 7 - All versus one, players uniting against one
    \item 8 - Ally with another player
    \item 9 - Pursue another player
    \item 10 - Survive, play to evade elimination
    \item 11 - Chain automatic effects (combo)
    \item 12 - Reduce or render useless a resource
    \item 13 - Defend from a player using another one
    \item 14 - Parallax: Change a point of view to create a better result for himself
    \item 15 - Avoid to acquire points to get some other benefit (playing early next round)
    \item 16 - Camping, in a position or determinate action
    \item 17 - Stop the progress of the game, delay the endgame
    \item 18 - Play accounting to your next actions, plan a series of actions.
    \item 19 - Alpha player: Player that forces the others to play as he wants
    \item 20 - Protect an specific position or pieces
    \item 21 - Play safely, do not take on risks only play on certainty
    \item 22 - Rituality: Repeat the same way of playing looking for the same results as before
    \item 23 - Counting cards, tokens and other resources
    \item 24 - Play riskily, pursue greater risks for greater gain
    \item 25 - Pursue self defined objective instead of the game objective
    \item 26 - Accelerate the end of the game
    \item 27 - Do a sacrifice for greater gains
    \item 28 - Distraction: Use an action to change other players attention from your real intention or objective
    \item 29 - Change strategy because of the game state
    \item 30 - Deduce hidden information through open information
    \item 31 - Communicate with allies indirectly so other players does not notice
    \item 32 - Forfeit the game
    \item 33 - Cheating: break the game rules
    \item 34 - Trolling: play to annoy other player rather than win the game
    \item 35 - Intimidate: use a stronger position to force another player to play as you want
    \item 36 - Bluffing: relay false information to manipulate other players actions
    \item 37 - Convince the other players
    \item 38 - Confound a player to induce a bad play
    \item 39 - Excluding another player
    \item 40 - Small talk: talking all the time to distract other players
 \end{itemize}
\end{multicols}

\subsection{Cruious cases}

Some concepts which arose during the focus groups discussions were very interesting although they did not reached the final list. This was because they were very complicated cases. But the fact that they were brought to discussion, is by itself a great contribution. 

One was Chaos, meaning a condition of disorder in the
game, that is, chaos made by the players. It was very
surprising to most, and through discussion they couldn’t
agree whether it is or not a dynamic. The most interesting
factor on this discussion, however, is that if it is a dynamic,
other possible conditions, like funny, tense and ordered,
would also be dynamics. On the other hand, it is also
possible that those concepts are actually part of Aesthetics,
as defined in the MDA\cite{Hunicke2004} since they could be associated to
emotional states.

Another case is the change of rules during the game.
This situation can happen in gameplay in some situations,
more notably when players notice a confusion or wrong
interpretation of the rules. It could also happen when players
find something broken while creating games, and it is
common on children’s play. The first case always rise the
question “does we keep playing wrong to avoid changing
the game as it is now, or from now on we play correctly?”.
Hence the possibility of being a dynamic, since it comes
from the decisions the players make. However, since it is
not part of the gameplay itself, it can be considered a meta-game
event.

The focus group also detected another common situation
in boardgames that can be considered a meta-game action:
a player needing or wanting to undo his last move or play,
including the case that it was impossible to be made with
the correct rules. 

All of those situations are left unanswered in this work.
They push way through the boundaries of our scope of
preliminary work, requiring more study and thought, and
were left for future development.