\section{Survey data}

Answers to the survey were collected for one week time, from 12/13/2018 to 12/20/2018. Featuring a total of 196 answers, which were divided in two groups: consumers with 174 replies and industry totalling 22 answers. Then the average value for each question in each group was computed. Those values are represented in \autoref{tab:avgResult}. The number of the question is the order in which it appears in the survey.

\begin{table}
    \centering
        \caption{Averages of answers}
    \begin{tabular}{c|c c|c c}
        Question & Hobbyists & stdev & Professionals & stdev\\
        1 & 4,07 & 1,05 & 4,23 & 0,92\\
        2 & 3,90 & 1,08 & 3,82 & 0,91\\
        3 & 3,92 & 1,11 & 4,36 & 0,73\\
        4 & 4,16 & 1,12 & 4,32 & 0,89\\
        5 & 4,04& 1,14 & 4,50 & 0,74\\
        6 & 3,21& 1,33 & 3,50 & 1,47\\
        7 & 3,35& 1,48 & 3,68 & 1,32\\
        8 & 3,72& 1,21 & 3,73 & 1,08\\
        9 & 2,62& 1,48 & 2,55 & 1,65\\
        10 & 3,48& 1,27 & 3,50 & 1,30\\
        11 & 4,16& 0,98 & 3,86 & 1,28\\
        12 & 3,41& 1,30 & 3,64 & 1,00\\
        13 & 3,17& 1,31 & 3,18 & 1,47\\
        14 & 3,57& 1,19 & 4,05 & 0,90\\
        15 & 3,75& 1,21 & 4,00 & 0,98\\
        16 & 3,01& 1,35 & 3,27 & 1,35\\
        17 & 2,94& 1,47 & 2,55 & 1,53\\
        18 & 4,44& 0,87 & 4,55 & 0,80\\
        19 & 2,68& 1,49 & 2,86 & 1,64\\
        20 & 3,93& 0,99 & 4,00 & 0,98\\
        21 & 3,46& 1,17 & 3,45 & 1,18\\
        22* & 2,88& 1,32 & 3,09 & 1,23\\
        23 & 3,75& 1,28 & 3,68 & 1,21\\
        24 & 3,07& 1,41 & 3,32 & 1,21\\
        25 & 3,78& 1,19 & 4,00 & 1,02\\
        26 & 4,06& 1,02 & 4,32 & 0,65\\
        27 & 4,11& 1,02 & 4,45 & 0,74\\
        28 & 4,34& 0,86 & 4,41 & 0,80\\
        29 & 4,06& 1,00 & 4,14 & 1,13\\
        30* & 2,63& 1,41 & 3,14 & 1,49\\
        31 & 1,88& 1,24 & 2,18 & 1,47\\
        32 & 1,76& 1,19 & 2,09 & 1,54\\ 
        33 & 2,27& 1,36 & 2,50 & 1,60\\
        34 & 3,06& 1,45 & 3,14 & 1,21\\
        35 & 3,71& 1,22 & 4,00 & 1,11\\
        36 & 3,82& 1,04 & 3,82 & 0,96\\
        37* & 2,93& 1,47 & 3,00 & 1,48\\
        38 & 2,07& 1,34 & 2,18 & 1,47\\
        39 & 2,26& 1,26 & 2,27 & 1,45\\
        40 & 3,91& 1,01 & 3,77 & 1,27
    \end{tabular}
    \caption*{* - differences between hobbyists and professionals}
    \caption*{stdev = standard deviation}
    \label{tab:avgResult}
\end{table}

Looking for the concepts which were mostly agreed by the subjects, that is, with average equal or higher then 3, becomes possible to observe an interesting phenomena. The industry group had 32 of the concepts in this level, on the other hand the consumers group had 29. It is important to note that the 29 of the consumers are present in the 32 from industry. This is, the consumers disagree only on three concepts from the industry, which lead to the belief that their opinions are similar or equal. Disagreeing with our previous assumption that they would have different opinions. Analyzing the concepts with even more average, those with a value greater or equal than 4, there is almost the same situation. With 14 concepts for industry and 9 for consumers we note a bigger difference in the quantity of higher values. Also of the consumers' 9, only one of them is not contained in the 14 from industry. Although it still poses a good evidence that the groups opinion are in accord, it also point some doubt in this assertion. To enlighten this doubt this work uses the ANOVA test to state if the groups have significantly different results.

