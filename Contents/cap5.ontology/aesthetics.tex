\section{OntoBG-A : aesthetics unearthed}

This ontology was a composition of two emotion models. Thus we needed to translate them to UFO patterns before uniting them, using a similar procedure from the mechanics ontology of \cite{kritz_buildingOntology} to achieve this translation. Finally their union came through the causal relationship established by this work. The diagram for this ontology can be found in Appendix \autoref{appendix:a5}

\subsection{Ekman to UFO}
Ekman's understanding of emotions was not stated as a formal conceptual model. But through careful study of his ideas' structure it is possible to create a conceptualization of emotions. His definitions and relationships proved to be well aligned with UFO modeling perspectives. Although such relationships are not explicitly defined in his work, it is possible to identify them from ideas and build them in the ontology.

Emotions are kinds. Following the ideas and definitions of Ekman's research it is easy to see that they are concepts which provide individuation and identification. This is also in accordance to MDA's model definitions and conceptualization. Which brings greater consistency to the ontology.

The atlas of emotion provides a spectrum of emotional states. These emotional states represent the different forms of experiencing the emotion. Even though linked to the scale of intensity of the emotion, which would induce modeling them as qualities, we chose to model them as subkinds of the emotion. This is due to the flexibility in using them directly to model games. Defining them as universals is important to recognize them more explicitly in the gameplay. This distinction makes possible to model a dynamic relating it directly to an emotional state, without the need to associate the emotion. For a game designer it brings value when specifying the outcome of a given dynamic. For example, it is better to say that Protectionism dynamic instigates a Pride emotional state. When saying it instigates Enjoyment emotion could include Schadenfreude emotional state in its possible outcomes, which is undesired when modeling for game design.

Compound emotions are acknowledged in Ekman's theory. But he does not provide a complete explanation of them, neither enough examples or structure. Even though, thinking of modeling games compound emotions might be of use to some games or game designers. Thus we define and include them in the conceptual model in resonance with the theory ideas.

\subsection{Dillon to UFO}
The 6-11 model presents two primary concepts: emotions and instincts. Emotions were already covered in Ekman's model, and Dillon seems to have emotions which are contained in it. Thus we focus on the instincts and its relationships.

Instincts were stereotyped as Kinds, following the same principles as the emotions. Their definitions and meanings in psychology theories bring them the same light of individuation and identification as emotions.

Different from Ekman, Dillon explicitly stated the relationships of his model. Featuring relationships between both instinct and emotions as well as inside the same group. This proved to be a challenge when trying to model them into the ontology. Using the Causal relationship created here we could imprint the meaning desired by Dillon in both associations alike. Thus they were stereotyped as Causal relationships, which also kept the directional propriety.

Differentiation of each of those relationships were done through their names. Between instincts there is an association of reinforcement thus the name Reinforces. Instincts are auxiliary in creating the feeling of emotions, thus an instinct Evoke an emotion. Emotions on the other hand pave the ground for instincts, which led to the Stimulate relationship.