\section{What about cause and consequence?}

Considering how different ideas relate to each other, MDA model comprise a notion of cause and consequence between its concepts. Stating that each part of the game spectrum is cause and consequence of each other. Dillon also uses this relationship to explain how emotions and instincts relate to each other. Even when addressing the interconnection between two instincts he utilize of cause and consequences.

Important on this type of relationship is that it have a reading direction. This is, when two concepts are related by cause and consequence. One of them is cause and the other is consequence and the cause brings about the consequence, never the other way around. Both models aforementioned clearly utilizes of this propriety. 

Although being directional, it does not mean the reflexive cannot exists. But when it does exist it does so with a different meaning. MDA uses it to differentiate its model reading direction. Even attributing each reading direction to different characters, one for the player of the game and other to the game designer. Dillon have few examples where the reflexive exists. He does not specifically state the difference in them. But in his comprehensive model explanation the difference in meaning is clear.

UFO does not have a theoretical basis for cause and consequence relationships. When looking to its relationship stereotypes, none of them quite allow the introduction of this relation. Although it could be stated as a simple association, its importance to each of the models implies necessity of a stronger theoretical basis. OntoBG absorb this importance from the participation of these models. Moreover its greatest contribution is to establish the relation between mechanics, dynamics and aesthetic concepts directly. Hence this work need the structuring of a precise relation stereotype for cause and consequence.

\subsection{Formalizing cause and consequence relationship}

Making a formal introduction requires first an analysis of this relation in our context. The prime sources of meaning are the 6-11 and MDA models. As this works is translating their respective conceptualizations to the UFO modeling language it is important to maintain the original meaning intended for this relationship. Luckily both models introduce cause and consequence very clearly in their structure. \todo{Reler dillon e MDA e citar as passagens de causa e consequencia. Eplicitando o sentido e cruzar as informações}