\section{What about cause and consequence?}

Considering how different ideas relate to each other, the MDA model comprises a notion of cause and consequence between its concepts, stating that each part of the game spectrum is cause and consequence of each other. \citet{dillon_way_2010} also uses this relationship to explain how emotions and instincts relate to each other, even when addressing the interconnection between two instincts, or two emotions.

It is important that this relationship has a reading direction. That is, when two concepts are related by cause and consequence, one of them is cause and the other is consequence, and the cause brings about the consequence, never the other way around. Both aforementioned models clearly use this propriety. 

However, even though it is directional, it does not mean that a different relationship in the reverse direction cannot exist. But, when it does exist, it does so with a different meaning. MDA uses it to differentiate its model reading direction, even attributing each reading direction to different characters, one for the player of the game and the other to the game designer. Dillon have few examples where the inverse exists. He does not specifically state the difference in them, but in his comprehensive model explanation the difference in meaning is clear.

UFO does not have a theoretical basis for cause and consequence relationships. When looking to its relationship stereotypes, none of them quite allow the introduction of this relationships. Although it could be stated as a simple association, its importance to each of the models implies necessity of a stronger theoretical basis. OntoBG absorb this importance from the participation of these models. Moreover, its greatest contribution is to establish the relationships between mechanics, dynamics and aesthetic concepts directly. Hence, this work needs the structuring of a precise relationships stereotype for cause and consequence.

\subsection{Formalizing cause and consequence relationship}


Making a formal introduction to the cause and consequence relationship first requires an analysis of this idea in our context. The prime sources of meaning are the 6-11 and MDA models. As this dissertation is translating their respective conceptualizations to the UFO modeling language, it is important to maintain the original meaning intended for this relationship. Luckily both models introduce cause and consequence very clearly in their structure. 

Dillon explains his model very briefly, not addressing the details and formalities of definitions, but he does make his ideas clear through examples. In those examples lie
 his understanding of relationships between his concepts of emotions and instincts. With this, creating a formal definition for such relationships should also drawn from those examples. Dillon construct his connections between concepts as a sequence, stating that the presence of a given instinct or emotion results in another one that by itself bring another, and so on. The idea of sequence of events is naturally linked to cause and consequence. He even reinforces this using phrases like ``survival instinct will provide new excitement''~\citep[p. 13]{dillon_way_2010} or ``It[Aggressiveness] can be triggered by survival and greed instincts''~\citep[p. 14]{dillon_way_2010}. Provide and trigger are words which convey the meaning of cause and consequence, e.g., trigger as verb explicitly means to cause something. When explaining his model, \citet{dillon_way_2010} mostly uses these constructions and thus heavily conveys  the idea that emotions and instincts are entwined by cause and consequences. 


MDA greatest distinction is its double sided view. The designer's perspective which goes from left to right and the player's one following the opposite way. When describing the difference in both views the authors make this statement: 

\begin{citacao}
From the designer's perspective, the mechanics give rise to
dynamic system behavior, which in turn leads to particular
aesthetic experiences. From the player's perspective,
aesthetics set the tone, which is born out in observable
dynamics and eventually, operable mechanics.
\cite{Hunicke2004}
\end{citacao}


For the designer's perspective they use an expression, ``give rise'', and a word, ``leads''. Both of them have cause and consequence meaning. The other way around is less explicit. From aesthetics to dynamics they employ ``born out'' when addressing the \textit{tone}, a consequence, of aesthetics. This expressions have the meaning that the observable dynamics comes from the tone. Which in turn is made by aesthetics, thus aesthetics become a cause of dynamics. 

Dynamics to mechanics becomes more complicated. The MDA description does not explicitly state this relationship. They envelop it as a part of the first relationship. Throughout their work, dynamics are more discussed with respect to aesthetics than mechanics. I believe this comes from the focus on the experience part of the game in the model, but there is something to be harnessed from their description: that mechanics become operable. What this means is, although not directly, dynamics influence mechanics, not their essence but their perceptiveness. Dynamics that happen during a game influence how the players perceive their possibilities of interaction with the game. That is, imagine a player that experiences a dynamic that arise from a different use of a given mechanic. If that situation provide different results from the traditional use of this mechanic, he surely will change the way he uses this mechanic in the future. In other words, the dynamics allows the player to understand how to operate the mechanics of the game, hence we found a causal relationship from dynamics to mechanics.
