\section{OntoBG-D : dynamics uncovered}

Structuring the concepts of dynamics that were created was the first step to create the ontology. To do this, we adopted a procedure. Firstly group the concepts by similarities using a \textit{Category} universal which generalizes them, which provide a taxonomic structure to dynamics. This process resulted in the following categories:
\begin{itemize}
    \item Action based: generalizes the dynamics which are based on the agency of the players.
    \item Intention of use: generalizes the dynamics which arise from a specific intention when using an action.
    \item Meta-game: generalizes the dynamics that happen outside of the game space but still inside the magic circle.
    \item Behaviour: generalizes the dynamics that represent a particular behaviour a player can adopt.
    \item Playing patterns: generalizes the dynamics which amount to how a player play the game along some time or even the whole game.
    \item Strategy choices: generalizes the dynamics that evaluate a specific play or principle used momentarily during the game.
\end{itemize}

After that, the 29 concepts of dynamics created in Chapter 4 were organized under these categories    . Some of the concepts did not fit in any of them. They also did not had enough similarities with the other concepts to justify a category, leaving them without generalizing categories, but still providing other types of relationships with other dynamics.

Afterwards, the dynamics were connected through part of relationships and generalizations. This was done using the discussions made during the focus group and the author's expertise. 

The diagram for this ontology can be found in Appendix \autoref{appendix:a4}.

\subsection{The expected}

Dynamics, by definition, are the behaviour of mechanics through input of the players during the game. The unexpected or not planned dynamics are not the only type of dynamics. The former work done to compose OntoBG-D focused on acquiring insight on those unexpected forms of dynamics. But only because they are the most tricky part of game design. The expected use and behaviour of game mechanics are also dynamics in this definition. Thus it needs to be included in this model, although not specifically as a dynamic for each mechanic expected use. This work is not comprehensive so it would be taxing to demand that for each mechanic that has an expected use a new dynamic should be created. Thus, expected use of dynamics needs to be a single concept.

Expected dynamics, in its essence are actually derived from their underlying mechanic. Firstly, one should notice that understanding them as the expected behaviour of a given mechanic makes it exclusively related to this mechanic. Imagine that we define the \textit{attack to damage} dynamic, which comes from the expected use of attacking to reduce life from an enemy. This dynamic cannot be related to a different mechanic, \textit{dashing into an enemy to damage it} is not \textit{attacking to reduce its life}. Being intrinsically related and attached to their underlying mechanics, the expected dynamics do not provide the sense of identity needed for a sortal concept. Thus pairing it with the other dynamics in the same abstraction level would be wrong from a conceptualization point of view. But using the more generic concept of a derived dynamic, that possess the sense of identity, fulfills our need for the expected behaviour of mechanics. Thus the \textit{kind} Derived, is added to the model. 

