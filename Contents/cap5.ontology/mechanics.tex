\section{OntoBG-M : mechanics unbound}

OntoBG-M lean heavily in the ontology created in \cite{kritz_buildingOntology}, using the same concepts and definitions. It enhances the structure with UFO stereotypes both for universals and relationships.

First to be said, is that all of the concepts are kinds. This is due to the conceptual meaning of each in the original ontology. All of them have instances and provide individualization. They are actual mechanics, even if each has a different abstraction level. This difference is explicit in the generalization hierarchy of the model, which is the same presented on the original work.

The most important increment to the model is the presence of the meronimic relations provided by UFO. Being able to identify mechanics to be part of one another brought the possibility of uniting both sides of the mechanics ontology. Algorithms and Data Representations mechanics could not be precisely related in the former work. Introducing parts and wholes theory broke this limitation. Connecting them came in two different aspects, one is the essential necessity of one mechanic to another and the other the possible interaction when designing a mechanic on top of another one. The first brought structural solidity. It is represented mostly by limitations such as:

\begin{itemize}
    \item Card is a essential part of deck;
    \item Die is essential to dice rolling;
    \item Resources are essential to resource management;
    \item Pattern is essential to both pattern recognition and pattern building;
\end{itemize}

Although looking obvious they are important to establish the structure of a game within the whole ontology. The other side of the meronimic relations in mechanics is used to model the design of a given boardgame. This comes in form of mechanics which can be used as a building block for another, like:
\begin{itemize}
    \item Areas can be part of pick up and delivery. This differentiate when the pick up and delivery is represented by actually moving something from an area to other one from just collecting resources and using them to complete tasks.
    \item Cards can be part of card draft. This is important because it show that a card draft, even though the name, does not need to be made of cards. Its idea can be applied with tiles or even tokens and still have the same significance.
\end{itemize} 

Looking into this can improve the possibilities of game design. By understanding if a mechanic can be combined with another and create meaningful play one can start answering why this happens on some cases and not all of them. Even more, analyzing the combinations already created allows for innovation. An attempt to create a new interaction between mechanic, successful or not, create invaluable knowledge about mechanics.