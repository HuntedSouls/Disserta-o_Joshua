\section{OntoBG-M : mechanics unbound}

OntoBG-M leans heavily on the ontology created in \cite{kritz_buildingOntology}, using the same concepts and definitions. It enhances the structure with UFO stereotypes both for universals and relationships. The diagram for this ontology can be found in Appendix \autoref{appendix:a3}

Firstly, all  concepts in OntoBG are kinds. This is due to the conceptual meaning of each one in the original ontology. All of them have instances and provide individualization, and they are actual mechanics, even if each of them has a different abstraction level. This difference is explicit in the generalization hierarchy of the model, which is the same presented on the original work.


The most important increment to the model is the presence of the meronymic relationships provided by UFO. Being able to identify mechanics to be part of one another brought the possibility of linking both sides of the mechanics ontology. Algorithms and Data Representations mechanics could not be precisely related in \citet{kritz_buildingOntology}, and introducing a parts and wholes theory broke this limitation. Connecting them came in two different aspects: one is the essential necessity of one mechanic to another and the other is possible interaction when designing a mechanic on top of another one. The first brought structural solidity. It is represented mostly by limitations such as:

\begin{itemize}
    \item Card is an essential part of deck;
    \item Die is essential to dice rolling;
    \item Resources are essential to resource management;
    \item Pattern is essential to both pattern recognition and pattern building;
\end{itemize}

% XEXEO PAROU AQUI
Although they look obvious they are important to establish the structure of a game within the whole ontology. The other side of the meronymic relationships in mechanics is used to model the design of a given board game. This comes in form of mechanics which can be used as a building block for another, for example:
\begin{itemize}
    \item Areas can be part of pick up and delivery. This differentiate when the pick up and delivery is represented by actually moving something from an area to other one from just collecting resources and using them to complete tasks.
    \item Cards can be part of card draft. This is important because it shows that a card draft, even though the name, does not need to be made of cards. Its idea can be applied with tiles or even tokens and still have the same significance.
\end{itemize} 

Looking into this can improve the possibilities of game design. By understanding if a mechanic can be combined with another and create meaningful play, one can start answering why this happens in some cases and not in all of them. Even more, analyzing the already created combinations allows for innovation. An attempt to create a new interaction between mechanics, successful or not, create invaluable knowledge about mechanics.