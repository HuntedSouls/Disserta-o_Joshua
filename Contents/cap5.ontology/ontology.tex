\chapter{OntoBG: An ontology of boardgames}

\todo[inline]{Esse texto está precisando de uma boa revisão, eu achei muito difícil de manter a leitura. Tá no nível C, sendo A inglês nativo, B inglês esperado por um não nativo e C tentando atingir o necessário, eu vou sair mudando o que já conseguir resolver}

This chapter covers how the ontology is, specifications, diagrams and restrictions. Intending to cover some important conceptual points inside the model. Also providing insights on how to use it for your own modeling purposes.

OntoBG, by what this works propose, is defined and explained here. Following the directives and scope it does not cover the whole of possibilities. It aims to provide new insight on boardgame modeling and study. 

The ontology is composed of four diagrams. A more general which express the MDA behaviour throughout the ontology. Three others which relay the understanding of each part of the MDA, that is, representing OntoBG-M, OntoBG-D and OntoBG-A. Following, each of them is presented and explained in a section. Finishing the chapter an overall look into the ontology is given. The last three diagrams are too big to fit inside the text and are present in [APENDIX X-> X+2].

Though, before understanding the model, there is one more concept to be created. The relationship of cause and consequence. Why this last creation was needed is fully explained in the first section.

\section{What about cause and consequence?}

Considering how different ideas relate to each other, MDA model comprise a notion of cause and consequence between its concepts. Stating that each part of the game spectrum is cause and consequence of each other. Dillon also uses this relationship to explain how emotions and instincts relate to each other. Even when addressing the interconnection between two instincts he utilize of cause and consequences.

Important on this type of relationship is that it have a reading direction. This is, when two concepts are related by cause and consequence. One of them is cause and the other is consequence and the cause brings about the consequence, never the other way around. Both models aforementioned clearly utilizes of this propriety. 

Although being directional, it does not mean the reflexive cannot exists. But when it does exist it does so with a different meaning. MDA uses it to differentiate its model reading direction. Even attributing each reading direction to different characters, one for the player of the game and other to the game designer. Dillon have few examples where the reflexive exists. He does not specifically state the difference in them. But in his comprehensive model explanation the difference in meaning is clear.

UFO does not have a theoretical basis for cause and consequence relationships. When looking to its relationship stereotypes, none of them quite allow the introduction of this relation. Although it could be stated as a simple association, its importance to each of the models implies necessity of a stronger theoretical basis. OntoBG absorb this importance from the participation of these models. Moreover its greatest contribution is to establish the relation between mechanics, dynamics and aesthetic concepts directly. Hence this work need the structuring of a precise relation stereotype for cause and consequence.

\subsection{Formalizing cause and consequence relationship}

Making a formal introduction requires first an analysis of this relation in our context. The prime sources of meaning are the 6-11 and MDA models. As this works is translating their respective conceptualizations to the UFO modeling language it is important to maintain the original meaning intended for this relationship. Luckily both models introduce cause and consequence very clearly in their structure. \todo{Reler dillon e MDA e citar as passagens de causa e consequencia. Eplicitando o sentido e cruzar as informações}
\section{It lives!}


\todo[inline]{Parece propaganda }
Translating the MDA framework into UFO led to a simple diagram with important information. Although simplistic, MDA has a lot of nuances. When addressed in a more complex modeling language this nuances need careful analysis to be correctly modeled. The diagram features in \autoref{fig:gamediagram} and its information detailed in sequence.

\begin{figure}[!h]
    \centering
    \includegraphics[scale=0.65]{Images/Model/Game.png}
    \caption{Game Diagram}
    \label{fig:gamediagram}
\end{figure}

This diagram illustrates the main idea of the MDA. The game is made of three different parts, mechanics, dynamics and aesthetics. They are inseparable and essential because a game must have all of them and they are intrinsic parts of games.

Addressing the designer and player different perspectives in MDA, the model includes the causal relationship between mechanics, dynamics and aesthetics. They were established in accordance to the discussion made in section 5.1.1.

\section{OntoBG-M : mechanics unbound}

OntoBG-M leans heavily on the ontology created in \cite{kritz_buildingOntology}, using the same concepts and definitions. It enhances the structure with UFO stereotypes both for universals and relationships. The diagram for this ontology can be found in Appendix \autoref{appendix:a3}

Firstly, all  concepts in OntoBG are kinds. This is due to the conceptual meaning of each one in the original ontology. All of them have instances and provide individualization, and they are actual mechanics, even if each of them has a different abstraction level. This difference is explicit in the generalization hierarchy of the model, which is the same presented on the original work.


The most important increment to the model is the presence of the meronymic relationships provided by UFO. Being able to identify mechanics to be part of one another brought the possibility of linking both sides of the mechanics ontology. Algorithms and Data Representations mechanics could not be precisely related in \citet{kritz_buildingOntology}, and introducing a parts and wholes theory broke this limitation. Connecting them came in two different aspects: one is the essential necessity of one mechanic to another and the other is possible interaction when designing a mechanic on top of another one. The first brought structural solidity. It is represented mostly by limitations such as:

\begin{itemize}
    \item Card is an essential part of deck;
    \item Die is essential to dice rolling;
    \item Resources are essential to resource management;
    \item Pattern is essential to both pattern recognition and pattern building;
\end{itemize}

% XEXEO PAROU AQUI
Although they look obvious they are important to establish the structure of a game within the whole ontology. The other side of the meronymic relationships in mechanics is used to model the design of a given board game. This comes in form of mechanics which can be used as a building block for another, for example:
\begin{itemize}
    \item Areas can be part of pick up and delivery. This differentiate when the pick up and delivery is represented by actually moving something from an area to other one from just collecting resources and using them to complete tasks.
    \item Cards can be part of card draft. This is important because it shows that a card draft, even though the name, does not need to be made of cards. Its idea can be applied with tiles or even tokens and still have the same significance.
\end{itemize} 

Looking into this can improve the possibilities of game design. By understanding if a mechanic can be combined with another and create meaningful play, one can start answering why this happens in some cases and not in all of them. Even more, analyzing the already created combinations allows for innovation. An attempt to create a new interaction between mechanics, successful or not, create invaluable knowledge about mechanics.
\section{OntoBG-D : dynamics uncovered}

Structuring the concepts of dynamics that were created was the first step to create the ontology. To do this we adopted a procedure. First group the concepts by similarities using a \textit{Category} universal which generalizes them. This provide a taxonomic structure to dynamics. Resulting in \todo{Colocar category escolhidas e falar sobre elas}


\section{OntoBG-A : aesthetics unearthed}

This ontology was a composition of two emotion models. Thus we needed to translate them to UFO patterns before uniting them. Using a similar procedure from the mechanics ontology of \cite{kritz_buildingOntology} to achieve this translation. Finally their union came through the causal relationship established by this work.

\subsection{Ekman to UFO}
Ekman understanding of emotions was not stated as a formal conceptual model. But through careful study of his ideas' structure it is possible to create a conceptualization of emotions. His definitions and relationships proved to be well aligned with UFO modeling perspectives. Although such relationships not being explicitly defined in his work, it is possible to identify them from ideas and build them in the ontology.

Emotions are kinds. Following the ideas and definitions of Ekman's research it is easy to see that they are concepts which provide individuation and identification. This is also in accordance to MDA's model definitions and conceptualization. Which brings greater consistency to the ontology.

Atlas of emotion provide a spectrum of emotional states. This emotional states represent the different forms of experiencing the emotion. Even though linked to the scale of intensity of the emotion, which would induce modeling them as qualities, we chose to model them as subkinds of the emotion. This is due to the flexibility in using them directly to model games. Defining them as universals is important to recognize them more explicitly in the gameplay. This distinction makes possible to model a dynamic relating it directly to an emotional state, without the need to associate the emotion. For a game designer it brings value when specifying the outcome of a given dynamic. For example, it is better to say that Protectionism dynamic instigates a Pride emotional state. When saying it instigates Enjoyment emotion could include Shadenfreud emotional state in its possible outcomes, which is undesired when modeling for game design.

Compound emotions are acknowledged in Ekman's theory. But he does not provide a complete explanation of them, neither enough examples or structure. Even though thinking of modeling games compound emotions might be of use to some games or game designers. Thus we define it and include in the conceptual model in resonance with the theory ideas.

\subsection{Dillon to UFO}
6-11 model presents two primary concepts, emotions and instincts. Emotions were already covered in Ekman's model, and Dillon seems to have emotions which are contained in it. Thus we focus on the instincts and its relationships.

Instincts were stereotyped as Kinds, following the same principles as the emotions. Their definitions and meanings in psychology theories bring them the same light of individuation and identification as emotions.

Different from Ekman, Dillon explicitly stated the relationships of his model. Featuring relationships between both instinct and emotions as well as inside the same group. This proved to be a challenge when trying to model them into the ontology. Using the Causal relationship created here we could imprint the meaning desired by Dillon in both associations alike. Thus they were stereotyped as Causal relationships, which also kept the directional propriety.

Differentiation of each of those relationships were done through their names. Between instincts there is an association of reinforcement thus the name Reinforces. Instincts are auxiliary in creating the feeling of emotions, thus an instinct Evoke an emotion. Emotions on the other hand pave the ground for instincts, which led to the Stimulate relationship.

