\chapter{OntoBG: An ontology of boardgames}

This chapter covers how the ontology is, specifications, diagrams and restrictions. Intending to cover some important conceptual points inside the model. Also providing insights on how to use it for your own modeling purposes.

OntoBG, by what this works propose, is defined and explained here. Following the directives and scope it does not cover the whole of possibilities. It aims to provide new insight on boardgame modeling and study. 

The ontology is composed of four diagrams. A more general which express the MDA behaviour throughout the ontology. Three others which relay the understanding of each part of the MDA, that is, representing OntoBG-M, OntoBG-D and OntoBG-A. Following, each of them is presented and explained in a section. Finishing the chapter an overall look into the ontology is given. The last three diagrams are too big to fit inside the text and are present in [APENDIX X-> X+2].

Though, before understanding the model, there is one more concept to be created. The relationship of cause and consequence. Why this last creation was needed is fully explained in the first section.

\section{What about cause and consequence?}

Considering how different ideas relate to each other, the MDA model comprises a notion of cause and consequence between its concepts, stating that each part of the game spectrum is cause and consequence of each other. \citet{dillon_way_2010} also uses this relationship to explain how emotions and instincts relate to each other, even when addressing the interconnection between two instincts, or two emotions.

It is important that this relationship has a reading direction. That is, when two concepts are related by cause and consequence, one of them is cause and the other is consequence, and the cause brings about the consequence, never the other way around. Both aforementioned models clearly use this propriety. 
However, even though it is directional, it does not mean that a different relationship in the reverse direction cannot exists. But, when it does exist, it does so with a different meaning. MDA uses it to differentiate its model reading direction, even attributing each reading direction to different characters, one for the player of the game and the other to the game designer. Dillon have few examples where the inverse exists. He does not specifically state the difference in them, but in his comprehensive model explanation the difference in meaning is clear.

UFO does not have a theoretical basis for cause and consequence relationships. When looking to its relationship stereotypes, none of them quite allow the introduction of this relationships. Although it could be stated as a simple association, its importance to each of the models implies necessity of a stronger theoretical basis. OntoBG absorb this importance from the participation of these models. Moreover, its greatest contribution is to establish the relationships between mechanics, dynamics and aesthetic concepts directly. Hence, this work needs the structuring of a precise relationships stereotype for cause and consequence.

\subsection{Formalizing cause and consequence relationship}


Making a formal introduction to the cause and consequence relationship first requires  an analysis of this idea in our context. The prime sources of meaning are the 6-11 and MDA models. As this dissertation is translating their respective conceptualizations to the UFO modeling language, it is important to maintain the original meaning intended for this relationship. Luckily both models introduce cause and consequence very clearly in their structure. 

Dillon explains his model very briefly, not addressing the details and formalities of definitions, but he does make his ideas clear through examples. In those examples lie
 his understanding of relationships between his concepts of emotions and instincts. With this, creating a formal definition for such relationships should also drawn from those examples. Dillon constructs his connections between concepts as a sequence, stating that the presence of a given instinct or emotion results in another one that by itself bring another, and so on. The idea of sequence of events is naturally linked to cause and consequence. He even reinforces this using phrases like ``survival instinct will provide new excitement''~\citep[p. 13]{dillon_way_2010} or ``It[Aggressiveness] can be triggered by survival and greed instincts''~\citep[p. 14]{dillon_way_2010}. Provide and trigger are words which convey the meaning of cause and consequence, e.g., trigger as verb explicitly means to cause something. When explaining his model, \citet{dillon_way_2010} mostly uses these constructions and thus heavily conveys  the idea that emotions and instincts are entwined by cause and consequences. 

MDA greatest distinction is its double sided view. The designer's perspective which goes from left to right and the player's one following the opposite way. When describing the difference in both views the authors make this statement: 

\begin{citacao}
From the designer's perspective, the mechanics give rise to
dynamic system behavior, which in turn leads to particular
aesthetic experiences. From the player's perspective,
aesthetics set the tone, which is born out in observable
dynamics and eventually, operable mechanics.
\cite{Hunicke2004}
\end{citacao}


For the designer's perspective they use an expression, ``give rise'', and a word, ``leads''. Both of them have cause and consequence meaning. The other way around is less explicit. From aesthetics to dynamics they employ ``born out'' when addressing the \textit{tone}, a consequence, of aesthetics. This expressions has the meaning that the observable dynamics comes from the tone. Which in turn is made by aesthetics, thus aesthetics become a cause of dynamics. 

Dynamics to mechanics becomes more complicated. The MDA description does not explicitly state this relationship. They envelop it as a part of the first relationship. Throughout their work, dynamics are more discussed with respect to aesthetics than mechanics. I believe this comes from the focus on the experience part of the game in the model, but there is something to be harnessed from their description: that mechanics become operable. What this means is, although not directly, dynamics influence mechanics, not their essence but their perceptiveness. Dynamics that happen during a game influence how the players perceive their possibilities of interaction with the game. That is, imagine a player that experience a dynamic that arise from a different use of a given mechanic. If that situation provide different results from the traditional use of this mechanic, he surely will change the way he uses this mechanic in the future. In other words, the dynamics allows the player to understand how to operate the mechanics of the game, hence we found a causal relationship from dynamics to mechanics.

\todo[inline]{Só para pensar, isso não tem relação com affordance. Discuta com o Mangeli}
\section{It lives!}

OntoBG, by what this works propose, is defined and explained here. Following the directives and scope it does not cover the whole of possibilities. It aims to provide new insight on boardgame modeling and study. 

The ontology is composed of four diagrams. A more general which express the MDA behaviour throughout the ontology. Three others which relay the understanding of each part of the MDA, that is, representing OntoBG-M, OntoBG-D and OntoBG-A. Following each of them is presented and explained in a section. Exposing how they came to be, their meaning and how to use each for your own purposes.  

\subsection{The beginning}

Translating the MDA framework into UFO led to a simple diagram with important information. Although simplistic MDA has a lot of nuances. When addressed in a more complex language this nuances need careful analysis to be correctly modeled. The diagram features in [FIGURE] and its information detailed in sequence.

[FIGURE]


\section{OntoBG-M : mechanics unbound}

OntoBG-M leans heavily on the ontology created in \cite{kritz_buildingOntology}, using the same concepts and definitions. It enhances the structure with UFO stereotypes both for universals and relationships. The diagram for this ontology can be found in Appendix \autoref{appendix:a3}

Firstly, all  concepts in OntoBG are kinds. This is due to the conceptual meaning of each one in the original ontology. All of them have instances and provide individualization, and they are actual mechanics, even if each of them has a different abstraction level. This difference is explicit in the generalization hierarchy of the model, which is the same presented on the original work.


The most important increment to the model is the presence of the meronymic relationships provided by UFO. Being able to identify mechanics to be part of one another brought the possibility of linking both sides of the mechanics ontology. Algorithms and Data Representations mechanics could not be precisely related in \citet{kritz_buildingOntology}, and introducing a parts and wholes theory broke this limitation. Connecting them came in two different aspects: one is the essential necessity of one mechanic to another and the other is possible interaction when designing a mechanic on top of another one. The first brought structural solidity. It is represented mostly by limitations such as:

\begin{itemize}
    \item Card is an essential part of deck;
    \item Die is essential to dice rolling;
    \item Resources are essential to resource management;
    \item Pattern is essential to both pattern recognition and pattern building;
\end{itemize}

% XEXEO PAROU AQUI
Although they look obvious they are important to establish the structure of a game within the whole ontology. The other side of the meronymic relationships in mechanics is used to model the design of a given board game. This comes in form of mechanics which can be used as a building block for another, for example:
\begin{itemize}
    \item Areas can be part of pick up and delivery. This differentiate when the pick up and delivery is represented by actually moving something from an area to other one from just collecting resources and using them to complete tasks.
    \item Cards can be part of card draft. This is important because it shows that a card draft, even though the name, does not need to be made of cards. Its idea can be applied with tiles or even tokens and still have the same significance.
\end{itemize} 

Looking into this can improve the possibilities of game design. By understanding if a mechanic can be combined with another and create meaningful play, one can start answering why this happens in some cases and not in all of them. Even more, analyzing the already created combinations allows for innovation. An attempt to create a new interaction between mechanics, successful or not, create invaluable knowledge about mechanics.
\section{OntoBG-D : dynamics uncovered}

Structuring the concepts of dynamics that were created was the first step to create the ontology. To do this we adopted a procedure. First group the concepts by similarities using a \textit{Category} universal which generalizes them. Which provide a taxonomic structure to dynamics. This process resulted in these categories:
\begin{itemize}
    \item Action based: generalizes the dynamics which are based on the agency of the players.
    \item Intention of use: generalizes the dynamics which arise from a specific intention when using an action.
    \item Meta-game: generalizes the dynamics that happens outside of the game space but still inside the magic circle.
    \item Behaviour: generalizes the dynamics that represent a particular behaviour a player can adopt.
    \item Playing patterns: generalizes the dynamics which amounts to how a player play the game along some time or even the whole game.
    \item Strategy choices: generalizes the dynamics that evaluate an specific play or principle used momentarily during the game.
\end{itemize}

Then the 29 concepts of dynamics created in Chapter 4 were organized under them. Some of the concepts did not fit in any of them. They also did not had enough similarities with the other concepts to justify a category. Leaving them without generalizing categories, but still providing other types of relations with other dynamics.

Afterwards the dynamics were connected through part of relationships and generalizations. This was done using the discussions made during the focus group and the author's expertise. 

\subsection{The expected}

Dynamics, by definition, are the behaviour of mechanics through input of the players during the game. The unexpected or not planned dynamics are not the only type of dynamics. The former work done to compose OntoBG-D focused on acquiring insight on those unexpected forms of dynamics. But only because they are the most tricky part of game design. The expected use and behaviour of game mechanics are also dynamics in this definition. Thus it need to be included in this model, although not specifically as a dynamics for each mechanic expected use. This work is not comprehensive so it would be taxing to demand that for each mechanic that has an expected use a new dynamic should be created. Expected use of dynamics need to be a single concept.

Expected dynamics, in its essence are actually derived from their underlying mechanic. First on should notice that understanding them as the expected behaviour of a given mechanic makes it exclusively related to this mechanic. Imagine that we define the \textit{attack to damage} dynamic, which comes from the expected use of attacking to reduce life from an enemy. This dynamic cannot be related to a different mechanic, dashing into an enemy to damage it is not attacking to reduce its life. Being intrinsic related and attached to their underlying mechanics the expected dynamics they do not provide the sense of identity needed for a sortal concept. Thus pairing it with the other dynamics in the same abstraction level would be wrong from a conceptualization point of view. But using the more generic concept of a derived dynamic, that possess the sense of identity, fulfill our need for the expected behaviour of mechanics. Thus the \textit{kind} Derived, is added to the model. 


\section{OntoBG-A : aesthetics unearthed}

This ontology was a composition of two emotion models. Thus we needed to translate them to UFO patterns before uniting them. Using a similar procedure from the mechanics ontology of \cite{kritz_buildingOntology} to achieve this translation. Finally their union came through the causal relation established by this work.

\subsection{Ekman to UFO}
Ekman understanding of emotions was not stated as a formal conceptual model. But through careful study of his ideas' structure it is possible to create a conceptualization of emotions. His definitions and relations proved to be well aligned with UFO modeling perspectives. Although such relations not being explicitly defined in his work, it is possible to identify them from ideas and build them in the ontology.

Emotions are kinds. Following the ideas and definitions of Ekman's research it is easy to see that they are concepts which provide individuation and identification. This is also in accordance to MDA's model definitions and conceptualization. Which brings greater consistency to the ontology.

Atlas of emotion provide a spectrum of emotional states. This emotional states represent the different forms of experiencing the emotion. Even though linked to the scale of intensity of the emotion, which would induce modeling them as qualities, we chose to model them as subkinds of the emotion. This is due to the flexibility in using them directly to model games. Defining them as universals is important to recognize them more explicitly in the gameplay. This distinction makes possible to model a dynamic relating it directly to an emotional state, without the need to associate the emotion. For a game designer it brings value when specifying the outcome of a given dynamic. For example, it is better to say that Protectionism dynamic instigates a Pride emotional state. When saying it instigates Enjoyment emotion could include Shadenfreud emotional state in its possible outcomes, which is undesired when modeling for game design.

Compound emotions are acknowledged in Ekman's theory. But he does not provide a complete explanation of them, neither enough examples or structure. Even though thinking of modeling games compound emotions might be of use to some games or game designers. Thus we define it and include in the conceptual model in resonance with the theory ideas.

\subsection{Dillon to UFO}
6-11 model presents two primary concepts, emotions and instincts. Emotions were already covered in Ekman's model, and Dillon seems to have emotions which are contained in it. Thus we focus on the instincts and its relationships.

Instincts were stereotyped as Kinds, following the same principles as the emotions. Their definitions and meanings in psychology theories bring them the same light of individuation and identification as emotions.

Different from Ekman, Dillon explicitly stated the relationships of his model. Featuring relations between both instinct and emotions as well as inside the same group. This proved to be a challenge when trying to model them into the ontology. Using the Causal relation created here we could imprint the meaning desired by Dillon in both associations alike. Thus they were stereotyped as Causal relations, which also kept the directional propriety.

Differentiation of each of those relations were done through their names. Between instincts there is an association of reinforcement thus the name Reinforces. Instincts are auxiliary in creating the feeling of emotions, thus an instinct Evoke an emotion. Emotions on the other hand pave the ground for instincts, which led to the Stimulate relation.

