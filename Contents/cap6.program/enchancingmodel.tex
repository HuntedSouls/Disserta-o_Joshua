\section{Enhancing the model}

Ontologies define domains. A non-comprehensive ontology is like a house in construction, with only the walls. Such house can shield you from the wind but cannot protect from the rain. This incomplete ontology has its uses, but it does not achieve its full potential. It does not mean, though, that this work is not complete. Proposing a non-comprehensive ontology mean that it should be created collaboratively. Using crowd knowledge and its multiple points of view rather than a single opinion. This creates a knowledge base founded upon the ideas of the users rather than a specialist.

Contributions are necessary, but have to be made carefully. It is important that the model remain cohesive and correct. With this the contribution has to go through some evaluation, which will guarantee that it is really enhancing the model.

There are two different contributions: a new concept or a new relationship. Each has to be evaluated differently, but they are linked. Creating new mechanics, dynamics or aesthetics include creating new relationships. An isolated concept will be incorrect. With this, first we introduce how to create relationships and then cover the new concepts. 

Before addressing the specific cases, there are some directives that are common for all of them. First thing to be evaluated is the possibility of redundancies. This is, evaluate if there is some other concept or relationship that has the same significance. To known this the user has to compare the description of each idea with his contribution. And ask whether they can be used to represent his idea, even if having some difference in abstraction level. Also use this process to evaluate if the new idea is contained or is an specification or generalization of some part of the model.

Users going through this process should take into account that its objective is to answer the question: This new idea is really important for the model? Meaning he needs to focus in understanding if the model is improved with his addition. 

Investigating redundancies is more conceptual than mechanic. Thus the main research should be made in the model's descriptions rather than through the program. Even then there is some use for the program. It can be used to find objects related to a concept in the model, which can help the user to find some idea related to his contribution.

\subsection{Creating relationships}

Relationships in OntoBG are mainly of two types: between mechanics, dynamics and aesthetics, henceforth called cross-model relationships; As well as within each model, henceforth called in-model relationships. The first one is exclusively of causal stereotype. Although it has some differences in regard to which types of concepts it connects. The other relationships can be generalizations, causal and meronimic relationships. Each has its own restrictions that need to be addressed. How to deal with each case is discussed here.

Firstly, the meaning of the relationships is of utmost importance. Cross-model relationships are necessarily causal. When evaluating this type of new relationship the user has to be sure that it has the semantics of a causal relationship. That is, if it fits the idea behind a cause and consequences established in chapter 5. Moreover, this relationship is connecting two concepts, and these concepts imply different semantic to the relationship. This is noted in the game model, that gives an abstract definition for these relationships. Connections between different concepts have slightly different ideas, which need to be present in the proposed relationship.

To include these relationships into the program the user creates the predicate representing his relationship. The different types of cross-model relationships are divided into two functors in the program.
\begin{itemize}
    \item mdRelation(MECHANIC,DYNAMIC) - Represents relationships between mechanics and dynamics, where MECHANIC and DYNAMIC are respectively the mechanic and dynamic of the relationship.
    \item daRelation(DYNAMIC,AESTHETIC) - Represents relationships between dynamics and aesthetics, where DYNAMIC and AESTHETIC are respectively the dynamic and aesthetic of the relationship.
\end{itemize}

When proposing in-model relationships the same principle holds. But each stereotype has its meaning, and thus the user can use this process to evaluate which is the stereotype of his new relationship. In-model causal relationships have to be in accord to cause and consequence, but do not have a more abstract definition like cross-model. Generalization relationships represent changes in abstraction level. Meronymic relationships are a bit more complicated and should be addressed carefully. Even so, most of the part of relationships will be quite intuitive, like \textit{card} being a part of \textit{deck}. Other cases might require the user to evaluate what meaning is behind his relationship to properly describe it and acknowledge if it fits the model.

Including these relationships in the prolog is done creating a predicate where the functor defines the stereotype of the relationship. Each is defined as:
\begin{itemize}
    \item generalization(A,B) - represent that A is a generic form of B, and conversely that B is more specific than A. 
    \item causal(A,B) - represent that A is the cause of the consequence B
    \item partof(A,B) - represent that A is a part of the whole B
\end{itemize}

Beware that each of these relationships defined are not reflexive. The order of the arguments change the meaning of the term. Due the conceptual relationship they represent being directional.

\subsection{Creating concepts}

Concepts provide a bit more challenge. This is because they need to have relationships to be in the model. When proposing a new concept the user must also propose relationships and go through all the process of assuring they should be included.

First step of the evaluation is to address which type of concept it is. To do this the user has to go through the definitions of each type. Taking into account how this new concept came to be, the user compares his idea with these definitions to establish its type. It may be tricky to differentiate dynamics from mechanics and thus, when proposing one of them, the user needs to be careful. All concepts in these models are \textit{kinds}. Their representation in the program is:
\begin{itemize}
    \item mechanics(NAME) - represents a concept which is a mechanic and belongs to its model. NAME is the name of the concept.
    \item dynamics(NAME) - represents a concept which is a dynamic and belongs to its model. NAME is the name of the concept.
    \item emotion(NAME) - represents a concept which is an emotion that is part of the aesthetic model. NAME is the name of the concept.
    \item instinct(NAME) - represents a concept which is an instinct that is part of the aesthetic model. NAME is the name of the concept.
\end{itemize}

Afterwards the user defines where this new concepts fit inside his type's model. Which means evaluating how it relates to the other concepts of the same type. This creates many relationships to be evaluated separately. Using their evaluation the user will be able to define how the new concept will relate to its peers. Although, it should be noted that it is not necessary to have these relationships. Concepts isolated from other ones of the same type is not incorrect. But as the ontology grows in number of concepts it will be harder for a concept not to relate with another. So it is better for the model quality if the concept has such relationships.

Cross-model relationships are now evaluated. They are necessary because of the MDA definition. The model then is only correct if a mechanic is related to a dynamic, dynamic to aesthetic and so on.

Including a new concept will demand creation of these relationships. To be able to identify these relationships in his concept, the user has to look into how his idea came to be. Normally this new concept is present in a game being created or analyzed by the user. He has to observe how this concept exists in this game. As example: if a mechanic, evaluate how it is used by players in different moments to decide which dynamics it creates. But the evaluation should not be limited to a single game. Looking into a diversity of games will provide better data and thus better relationships.

\subsection{Hit the nail in the head}

Having established the contribution correctness and its value, it is time to decide whether this is a good addition to the model. This process described should have given the user enough knowledge about his contribution and the model. With it he has the tools to decide.

To make this decision, the user analyzes his information. Focusing on the question: Does my contribution makes the model a better one? This means he should inquire whether the contribution is repetitive. Even though it is not redundant it can still be repetitive. That is, this new concept or relationship is used only in minor cases or is so close in meaning to an existing one that it is never used.

All these steps and difficulties in creating new objects for the model exist with a purpose. They ensure that the model does not become redundant or inexpressive. That is, having multiple concepts and relationships which are only simplifications of the other ones. This leads to a model which is large but cannot describe many aspects of games, being inexpressive. In other words, keep the model lean. 