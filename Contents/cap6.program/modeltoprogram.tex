\section{Translating OntoBG to Prolog}

Transporting the meaning comprised in a conceptual model to a different language can be a challenge. But prolog is first order logic, which is quite useful to describing concepts in a simple form. Following this idea of simplicity of traditional logic, this process focused in transferring the ideas with minimal modifications.

OntoBG concepts are divided in three types, mechanics, dynamics and aesthetics. Then the concepts inside each of them are introduced in prolog using their type as a functor and their own name as argument. This provides a clear distinction of the concepts and ease of access to the bigger group. In example: mechanics(action) represent the \textit{Kind} Action that is a mechanic, dynamics(actionBased) is the \textit{Category} Action Based that is a dynamic. Aesthetics has a distinction between its kinds because the model is composed of emotions and instincts. This is brought to the model as different functors, both representing concepts of aesthetic model. In example: emotion(fear) is the \textit{kind} Fear, the instinct(agressiveness) represents the \textit{kind} agressiveness. 

Relationships on the model are \textit{Causal}, Generalizations or meronymic. Causal is directly created in a functor with two arguments, where the first one is the cause and the second the consequence. In example: causal(instinct(identification),emotion(fear)). Generalization is also made with a functor with two arguments, like generalization(mechanic(phase),mechanic(ruleset)). But it has some restrictions. To be correct there can be no circular generalization. That is, if generalization(A,B) exists, generalization(B,A) cannot exist. Meronymic relationships has many forms, but in OntoBG they mostly are part-whole relationships. Thus we introduce it in prolog as a functor again, called \textit{partof}. The first argument is the part and the second the whole. Giving priority to the part over the whole comes from a characteristic of the model. Most of the part-whole relationships in the model are centered in the part role. If there is any interest in a whole focused relationship it should be added as a whole-of functor. 

All relationships defined are not reflexive. This means that the order of the arguments are important. A change in the order of the arguments creates a great semantic difference. Which means that most of the times using the inverted order is incorrect. 

Regarding other proprieties of the relationships each has its own rules. Generalization and part of is necessarily transitive on itself. Whilst causal might not be some times. Although when causal and part of have transitivity some times it does not hold the same meaning of the smaller relationships. But as the prolog model does not hold the specificity of the relationships, like cardinality and semantics. The program can define this transitivity directly without taking this into consideration.

Relationship between mechanics, dynamics and aesthetics are very important for the usability of the model. As they are responsible for describing games and defining their correctness. OntoBG presents these relationships with each direction having different meaning. But for the purpose of use of this program this is not relevant. As such they are defined in functors as:
\begin{itemize}
    \item mdRelation (MECHANIC,DYNAMIC) : - determines that there is a relationship between the MECHANIC and the DYNAMIC
    \item daRelation (DYNAMIC,AESTHETIC) : - determines that there is a relationship between the DYNAMIC and the AESTHETIC
\end{itemize}