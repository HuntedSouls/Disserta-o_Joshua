\section{Motivation}

The program is needed as a tool for using the ontology. The conceptual model and its specifications are useful by themselves when the objective is to learn and understand boardgames. For more practical applications a logical program that allows automated inquiries about the model becomes of utmost importance. This is the reason this work present a prolog program for navigation within the model.

The first part of the program will be the description of the ontology in a prolog database. First the concepts of the ontology will be described as a functor with a single argument. Then the relationships will be defined using compound terms.  Rules, which are used for more complex queries, are created focusing specific questions and the integrity of the model.

Pratical uses of the ontology mostly include studying specific games. Thus the program includes a compound term to define a game, in respect to their mechanics, dynamics and aesthetics. This in conjunction with the rules created for specific questions allows the user to fully use the complete potential of the model.