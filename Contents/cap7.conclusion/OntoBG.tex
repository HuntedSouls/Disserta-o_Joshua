\section{OntoBG: ontology of boardgames}

Here it was assembled a novelty approach to the study of boardgames: an ontology of the domain. It is not the only approach at the time of publishing as in July of 2019 the book \textit{Building blocks of tabletop game design} \cite{engelstein2019building} was released. It provides encyclopedic knowledge in mechanics of tabletop games. While it is a far more precise and comprehensive work in terms of mechanics, this work has a broader aspect when addressing dynamics and aesthetics. In \autoref{appendix:a6} there is a table with an analysis of how this new work should be evaluated in the light of this theory and ontology. Nonetheless, this work still provide different insight and usability in comparison to this book.

OntoBG at the end of this work is an ontology that represents the world of analog games. It does so using a heavy theoretical basis, which is one of the greatest contributions of this work. The MDA framework used as core for the ontology is a flexible and widely accepted model. This allows the creation of a more in depth theory of its aspects. The knowledge on mechanics, dynamics and aesthetics, specially the last two, are the greatest value of this work. 

This ontology was made to represent the knowledge of the players and designers of such games. But to better represent it they were crossed with academic knowledge. Mechanics were extracted from a widely used forum then structured and verified with academic researches in \cite{kritz_buildingOntology}. Dynamics were evaluated through a focus group and survey, then structured according to the author academic knowledge. For aesthetic aspects of the game, two models, one from a game designer and other from a psychologist were mixed to create better knowledge.

Using this ontology as basis it will be possible to create a more complete knowledge of games. Both scholar and industry shall find this mixed approach of great insight.

OntoBG is not comprehensive as it is and it is far from complete. Which leads to a clear future work of complementing it. This can be done by analysing other experiences and points of view to discover new concepts and relationships not featured in the model. Also, reviewing the ontology through other lenses, comparing it to other models or theoretical structures, aiming to find what it lacks and then fill the gaps are all possible contributions to be done.

This ontology was created mostly to be used in analysis and creation of boardgames. Including the ontology on game design methods, describing games and comparing different types of games are all good forms of using this work. This form of new work is very important. So far, there is a lack of structure in those processes for boardgames.

