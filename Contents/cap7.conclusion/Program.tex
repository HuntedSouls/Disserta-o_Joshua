\section{Program of the model}

Lastly this work implemented a program in the prolog language. This program translated the model and made it computable. It is responsible for most of the usability of OntoBG. Allowing an user to evaluate his own games and easily describe new ideas into the model. Creating a tool for interaction with the model is important to provide the means for achieving this works purpose. That is, to help designers and scholars a facilitating tool for their craft. Be it in describing or analyzing games.

There is room for improvements in the program. It does not automatically verify the possibility of a new idea. And if correct it still should be added manually rather than through the program execution. Other new functionalities would be new rules and processes to evaluate described games. Automatic generating games, given some parameters. Proposing changes in games as the user needs, for example suggesting a different set of mechanics that does not remove any dynamics.

Another interesting approach in this line would be to import descriptions of games from their manual. This would mostly generate mechanics and dynamics. Nonetheless it would be interesting to compare with human made descriptions of these games.