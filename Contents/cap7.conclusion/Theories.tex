
\section{Theories}

Not only the practical part of this work is a contribution. The theories established here provide new insight about boardgames. It is divided in three different areas, mechanics, dynamics and aesthetics. Each with a contribution, which is covered followingly.

\subsection{Preliminary work on dynamics}

Throughout the process of creating the theoretical background of the ontology. This work addressed dynamics as an academic subject in games. This by itself is a contribution. Dynamics are very neglected by both industry and scholars. And this work create a model of these dynamics, which is not seen in boardgame literature. Thus presenting a theoretical foundation which can be used to structure new research.

Proposing new works in dynamics shall begin with new research on the vocabulary created. It is clearly lacking and simple. Together with new researches in game dynamics subject the model can be greatly improved. Another good use for a model of dynamics is identifying them automatically from gameplay data. This can lead to new forms of classification for games as well as change how we analyze e-sports data. To create this automatization one should unite with studies on game telemetry, game provenance and process mining.

\subsection{Aesthetic values in perspective}

Aesthetic is a term originated from philosophy. With this is not actually clear of what it means. Specially for games, where there are two common interpretations, visual appreciation and emotions. MDA  framework \cite{Hunicke2004} agrees with the emotions as the aesthetic response that matter for games. But it might not be everything. With this in mind here was built a model for aesthetic responses including instincts and emotions.

Emotions and instincts were first related in the games perspective by Dillon \cite{dillon_way_2010}. His work was greate because it was created inside the games domain and based on a game designer experience. But given the fact that the understanding of what is aesthetics is by itself without common agreement. Here we looked upon a different area of knowledge for more information on what are emotions. Using Ekman's \cite{ekmans_atlas} theory we enhanced Dillon's perspective and created a reliable model for aesthetics of games.

Next steps in this line of research should first look into other areas for different perspectives on emotions and instincts. Which would enhance even more the theoretic foundations established in this work. In particular it would be of great interest to the game studies community to go through philosophy and fully understand aesthetics.

\subsection{Mechanics and praxis}

The mechanics model was created using a user created database. With this it reflects what these persons believed about games. When this knowledge was put through the lenses of other scholar's ideas it was clearly insufficient. This work did not fix this aspect of the model, but linked it to the other models created. Which by itself improves what is said of the mechanics

Creation of this ontology was detailed thoroughly in \cite{kritz_buildingOntology}. There it was already discussed the enormity of creating a good database on mechanics knowledge. This work did not increased the mechanics specific knowledge because it would rather cover more ground using more concepts of games. Leaving this effort to future work.

