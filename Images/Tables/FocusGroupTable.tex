\begin{table}[ht]
    \centering
    \resizebox{\columnwidth}{!}{
    \begin{tabular}{l}
Fazer uma ação para retirar opções dos outros jogadores & 
Usar efeito colateral, fazer A causa B que atinge C &
Usar uma ação para descobrir informações, ver reação dos outros jogadores. &
Bloquear outro jogador (cortar) &
Usar uma ação para mudar o estado do jogo(fase, estágio, turno) &
1 contra todos, um jogador decidir atacar todos os outros &
Todos contra 1, todos os jogadores se unirem contra um jogador &
Se aliar a outro jogador &
Perseguir um jogador &
Sobreviver, jogar para não ser eliminado & 
Encadear efeitos automáticos do jogo (combo) &
Inutilizar ou reduzir um recurso do jogo &
Se proteger de um jogador usando um terceiro jogador &
Paralaxe: Observar um resultado visando forçar um resultado mais benéfico para sí (daqui ta enconstando!) &
Evitar pontuar no jogo para obter algum ganho (não ser o último a jogar) &
Acampar em uma posição / fazer sempre uma mesma ação &
Impedir o progresso do jogo / atrasar o jogo &
Jogar em função das suas próximas ações/ planejar uma série de ações &
Alpha player: Jogador que joga pelos outros, que força outros jogadores a jogarem como ele quer que joguem. &
Proteger uma posição ou peças (jogar de forma a impedir que outros jogadores acessem a posição ou peça) &
Jogar de forma segura (não assumir riscos e jogar o máximo na certeza) &
Ritualidade: Jogador que repete a forma de uma jogada visando um mesmo resultado obtido antes. (jogar o dado de uma forma específica para sair melhor resultado) &
Contar cartas, tokens e outros recursos. &
Jogar de forma arriscada (visar os maiores riscos pelas melhores recompensas) &
Procurar atingir objetivo próprio diferente do definido pelo jogo (só conquistar áreas montanhosas, 'sou o rei da montanha') &
Acelerar o fim do jogo &
Fazer um sacrificio (sacrificar um peça no xadrez) &
Distração: fazer uma ação para tirar o foco dos outros jogadores de sua real intenção/objetivo &
Mudar de estratégia por causa do status do jogo (como não tem mais pontos por milho vou pontuar por madeira) &
Deduzir informações secretas através de informações abertas &
Se comunicar de forma indireta com aliados para que os outros não identifiquem (sussurrar, mimica, olhares penetrantes) &
Desistir do jogo (ficar apenas andando em circulos sem sentido algum) &
Trapacear: quebrar uma regra do jogo &
Trollar: jogar para atrapalhar os outros jogadores, sem se importar em vencer &
Intimidar: usar de uma posição de maior força para forçar uma jogada de outro jogador. &
Blefar: passar informações falsas para manipular a jogada dos outros jogadores &
Convencer os jogadores de algo (convencer os jogadores de que você não tem interesse em algo para que não lhe ataquem) &
Confundir um jogador para que ele faça uma jogada ruim &
Excluir alguém do jogo (em um jogo que se escolha participantes, nunca escolher um jogadore especifico) &
Small talk: falar durante todo o tempo para distrair outros jogadores 
    \end{tabular}}
    \caption{The 40 ideas defined in the focus group}
    \label{tab:focusgroupraw}
\end{table}